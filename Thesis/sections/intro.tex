As China's economic influence continues to grow, its lending practices to developing countries have come under scrutiny.
The concept of DTD (hereinafter DTD), whereby China extends excessive loans to countries in exchange for political or economic concessions, has become a topic of heated debate \citet{Chellaney_2017}.
While some argue that the debt-trap problem poses a significant threat to the economic and political stability of vulnerable countries, others contend that it is overstated.

The concept of DTD has been the subject of much debate in recent years.
\citet*{Parker2018} states that China DTD as a technique to achieve strategic objectives, such as projecting power across South Asian trading routes, undermining regional opposition to its South China Sea claims, and supporting its naval efforts to break out into the Pacific.
Critics of this phenomenon argue that claims of DTD are often exaggerated or based on incomplete information. For example, \citet*{Brautigam-meme-2020} argues that the debt-trap is based on a flawed understanding of Chinese lending practices and the histories of the target countries, and that China is not strategically pursuing the DTD on developing countries.

The opacity in Chinese lending practices has been a longstanding challenge in the analysis of the DTD problem.
The lack of transparency in the Chinese lending system, whereby loan terms, conditions and collateral requirements are not always disclosed to the borrowers, makes it difficult for economists and policymakers to fully grasp the magnitude of the issue.
Specifically, China's official external lending is predominantly undertaken by state-owned entities and the government itself\footnotemark{}. However, unlike other major economies, the Chinese government does not report or publish any data on its official international lending or outstanding overseas debt claims. This lack of transparency creates challenges for rating agencies, as official lending to sovereigns is not a regular part of their activities. Moreover, China is not a member of the Paris Club, which tracks sovereign borrowing from official bilateral creditors, and does not divulge data on its official flows with the OECD's Creditor Reporting System. Therefore, documentation of China's international lending has not been comprehensive, making it challenging to determine the true nature and extent of the debt-trap problem \citep*{Horn-Reinhart-Trebesch-21}.
With limited access to information on Chinese loans, it has been challenging to assess the sustainability of debts of borrowing countries and their ability to service their obligations, as well as the potential impact of China's lending practices on the economy of low income developing countries (LIDC), especially those in the Belt and Road Initiatives (BLI).
\footnotetext{These include China's state-owned policy banks, such as China Development Bank (國家開發銀行, CDB) and China
Export-Import Bank (中國進出口銀行, Ex-Im), as well as China's state-owned commercial banks such as Industrial and Commercial
Bank of China (中國工商銀行, ICBC) or Bank of China (中國銀行, BoC)}
As a result, recent studies on whether DTD is a myth have primarily been conducted normatively in the field of political science, rather than a positive economics analysis~\citep[See, e.g.,][]{Himmer2023-vn,Chen2020-eo}.

However, with the emergence of new and detailed data on Chinese lending practices, recent studies have begun to shed light on the nature and implications of the DTD.
In this thesis, I aim to shed light on this complex issue by applying a new and detailed data provided by \citet*{Horn-Reinhart-Trebesch-21}, hereinafter referred to the ``HRT database'', on the sovereign debt model proposed by \citet*{Na-18}, to provide insights into the sustainability of the debts of borrowing countries.
By calibrating the model for a particular country, a set of tradable-output levels which would cause the country to default could be obtained, given its current debt level. Following the approach of \citet{Hinrichsen_2020-chapter4}, this set is presented graphically with each data point on the space representing a debt-output pair for a specific year. This visual representation allows for an examination of whether the country has been in the default zone but has managed to avoid default due to other enforcement mechanisms, in this case the might be the political leverage from China.
