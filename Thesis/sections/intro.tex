As China's economic influence continues to grow, its lending practices to developing countries have come under scrutiny.
The concept of ``debt-trap diplomacy'', whereby China extends excessive loans to countries and places a debt burden upon them in exchange for political or economic concessions, has become a topic of heated debate \citep{Chellaney_2017}.

From the political science aspect, an analysis from the Belfer Center for Science and International Affairs states that China utilizes the debt-trap diplomacy as a technique to achieve strategic objectives, such as projecting power across South Asian trading routes, undermining regional opposition to its South China Sea claims, and supporting its naval efforts to break out into the Pacific \citep*{Parker2018}.
On the opposite side of the spectrum, critics of this phenomenon argue that claims of the debt-trap diplomacy are often exaggerated or based on incomplete information. For example, \citet*{Brautigam-meme-2020} notes that the debt-trap is based on a flawed understanding of Chinese lending practices and the histories of the target countries, and that China is not strategically pursuing the debt-trap diplomacy on developing countries.

Does the excessive debt to China cause a potential impact on the economies of low income developing countries (LIDC), especially those in the Belt and Road Initiatives (BRI)? Not many empirical studies examine this issue.
On one hand, the lack of transparency in the Chinese lending system, whereby loan terms, conditions, and collateral requirements are not always disclosed to the borrowers, makes it difficult for economists to fully grasp the magnitude of the issue.
On the other hand, the decision for a country to continue honoring debt obligations is typically an optimal decision in the absence of enforcement. Hence, it is not obvious whether the country is genuinely in good standings, or is it that the country would actually be better off if it were to default, but is somehow forced not to due to other enforcement. In this case it might be the political leverage from China.
As a result, recent studies on whether debt-trap diplomacy is a myth have primarily been conducted normatively in the field of political science, rather than a positive economics analysis~\citep[see, e.g.,][]{Himmer2023-vn,Chen2020-eo}.

In this thesis, I aim to shed light on whether a country falls into the debt-trap by using the sovereign debt model proposed by \citet{Na-18} and the graphical approach presented by \citet{Hinrichsen_2020-chapter4}, in order to provide insights into the sustainability of the debt of borrowing countries.
The concept of addressing the issue through the above methods follows \citet{Ho-23-debt-trap}.
By calibrating the model for a particular country, a set of tradable-output levels that would cause the country to default could be obtained, given its current debt level. The approach of \citet{Hinrichsen_2020-chapter4} then allows us to present the default set graphically, with each datapoint on the space representing a debt-output pair for a specific year. This visual representation allows for an examination of whether the country has been in the default zone, but has managed to avoid default due to other enforcement mechanisms.

% Conducting calibration on all countries to match the model requires enourmous amount of work, therefore it is optimal to narrow down the sample countries to those that provides the most insight on the DID issue. I consider countries
% \begin{enumerate*}[label = (\roman*)]
%     \item that are constantly receiving loans from other international institutes;
%     \item that indicate increasing amounts of debt from China that eventually exceed other creditors; and
%     \item in which China launches large infrastructure programs.
% \end{enumerate*}
The default sets for Sri Lanka and Pakistan are examined in this thesis. Sri Lanka is mentioned in the origin of the debt-trap diplomacy narrative \citep{Chellaney_2017}, and Pakistan, being the centerpiece of the China Pakistan Economic Corridor (CPEC), is also discussed frequently in the literature \citep{Hurley19-8-debt-trap}.

To my best knowledge, there is so far no empirical study on the issue of debt-trap diplomacy based on an economic model with microfoundation. My thesis contributes to the debt-trap diplomacy literature by conducting the first empirical result based on a sovereign debt model calibrated with the data of receiving countries under debate. The goal is to investigate whether the loans provided by China indeed push the countries to the brink of default and to report counterfactual results of China not lending the considerable amount of loans to BRI countries.

Through rigorous analysis and empirical evidence, it has been established that
Sri Lanka was indeed moving toward the default set after the intervention of China's excessive loans on large constructions in 2008, and Pakistan, though already in the crisis of defaulting in 2010, was pushed further by the substantial amount of loans on power projects.
Accordingly, I categorize two types of debt traps: the type-one debt trap, exemplified by Sri Lanka, where a stable financial state is disrupted by external intervention, and the type-two debt trap, exemplified by Pakistan, where an indebted country becomes even more insolvent as lending progresses.

The remaining chapters are organized as follows:
Chapter \ref{ch:lit} conducts a comprehensive literature review on the topics of debt-trap diplomacy and sovereign default models.
Chapter \ref{ch:data} briefly describes the characteristics of external debts to China, as well as the current situation of Sri Lanka and Pakistan.
Chapter \ref{ch:model} outlines the specific sovereign debt model that will be applied in this study.
Chapter \ref{ch:result} presents the calibration of the model and provides the corresponding empirical results.
Finally, Chapter \ref{ch:conclusion} concludes the thesis.


