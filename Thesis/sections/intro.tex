As China's economic influence continues to grow, its lending practices to developing countries have come under scrutiny.
The concept of ``debt-trap diplomacy'' (hereinafter DTD), whereby China extends excessive loans to countries in exchange for political or economic concessions, has become a topic of heated debate \citep{Chellaney_2017}.
While some argue that the debt-trap problem poses a significant threat to the economic and political stability of vulnerable countries, others contend that it is overstated.

From the political science aspect, an analysis from the Belfer Center for Science and International Affairs states that China utilizes DTD as a technique to achieve strategic objectives, such as projecting power across South Asian trading routes, undermining regional opposition to its South China Sea claims, and supporting its naval efforts to break out into the Pacific \citep*{Parker2018}.
Critics of this phenomenon argue that claims of DTD are often exaggerated or based on incomplete information. For example, \citet*{Brautigam-meme-2020} argues that the debt-trap is based on a flawed understanding of Chinese lending practices and the histories of the target countries, and that China is not strategically pursuing the DTD on developing countries.

The opacity in Chinese lending practices has been a longstanding challenge in the analysis of the DTD problem \citep*{Horn-Reinhart-Trebesch-21}.
The lack of transparency in the Chinese lending system, whereby loan terms, conditions and collateral requirements are not always disclosed to the borrowers, makes it difficult for economists and policymakers to fully grasp the magnitude of the issue.
Therefore, it has been challenging to assess the sustainability of debts of borrowing countries and their ability to service their obligations, as well as the potential impact of China's lending practices on the economy of low income developing countries (LIDC), especially those in the Belt and Road Initiatives (BLI).
As a result, recent studies on whether DTD is a myth have primarily been conducted normatively in the field of political science, rather than a positive economics analysis~\citep[See, e.g.,][]{Himmer2023-vn,Chen2020-eo}.

However, with the emergence of new and detailed data on Chinese lending practices, recent studies have begun to shed light on the nature and implications of the China debt.
In this thesis, I aim to shed light on whether a country fell into the debt-trap by applying a new and detailed data provided by \citet*{Horn-Reinhart-Trebesch-21}, hereinafter referred to the ``HRT database'', on the sovereign debt model proposed by \citet*{Na-18}, to provide insights into the sustainability of the debt of borrowing countries.
By calibrating the model for a particular country, a set of tradable-output levels which would cause the country to default could be obtained, given its current debt level. Following the approach of \citet{Hinrichsen_2020-chapter4}, this set is presented graphically with each data point on the space representing a debt-output pair for a specific year. This visual representation allows for an examination of whether the country has been in the default zone but has managed to avoid default due to other enforcement mechanisms, in this case might be the political leverage from China.

It requires a lot of work to calibrate all countries to match the model, therefore it is optimal to narrow down the sample countries to those that provides the most insight on the DID issue. I consider countries
\begin{enumerate*}
    \item that is constantly receiving loans from other international institutes,
    \item that indicates an increasing amount of debt from China that eventually exceeds other creditors, and
    \item in which China launches large infrastructure programs.
\end{enumerate*}
% \citet*{Hurley19-8-debt-trap} examines the dept-to-GDP ratio versus the share of China's debt, and identifies eight
% countries\footnote{
%     These countries are Djibouti, Kyrgyzstan, Laos, the Maldives, Mongolia, Montenegro, Pakistan, and Tajikistan}
% that are particularly risky.
In particular, the default sets for Sri Lanka and Pakistan are examined in this thesis. I will further provide the basic backgrounds for the two sample countries in later sections.

\section*{HRT Database}
International debt to China lacks transparency since its official lending is predominantly undertaken by state-owned entities. Moreover, unlike other major economies, the China government does not report or publish any data on its official international lending or outstanding overseas debt claims \citep*{Horn-Reinhart-Trebesch-21}.

% China's official external lending is predominantly undertaken by state-owned entities and the government itself\footnotemark{}. However, unlike other major economies, the Chinese government does not report or publish any data on its official international lending or outstanding overseas debt claims. This lack of transparency creates challenges for rating agencies, as official lending to sovereigns is not a regular part of their activities. Moreover, China is not a member of the Paris Club, which tracks sovereign borrowing from official bilateral creditors, and does not divulge data on its official flows with the OECD's Creditor Reporting System \citep*{Horn-Reinhart-Trebesch-21}.
% \footnotetext{These include China's state-owned policy banks, such as China Development Bank (國家開發銀行, CDB) and China Export-Import Bank (中國進出口銀行, Exim), as well as China's state-owned commercial banks such as Industrial and Commercial Bank of China (中國工商銀行, ICBC) or Bank of China (中國銀行, BoC)}

\citet*{Horn-Reinhart-Trebesch-21} combines a variety of sources to construct a consensus database of Chinese official loans.
Their database spans from 1949, the establishment of the People's Republic of China, to 2017. It contains a granular dataset of 2151 loans and 2824 grants with information such as creditor agent, borrower type, commitment, maturity, etc. It also provides an aggregate panel data of the external debt to China for each country.
To have a more precise data of the China debt issue, the database constructed by \citet*{Horn-Reinhart-Trebesch-21} is utilized as the primary source. This database is combined with debt from other creditors obtained from the International Debt Statistics (IDS) from the World Bank.

The top 30 countries with the largest debts to China are displayed in \autoref{fig:total-debt-30}. Notably, Russia owes China over \$70 billion, while Pakistan's debt amounts to \$27 billion, both topping the list. Brazil and Venezuela are among the top 10 countries with the highest debt to China in Latin America. Contrary to what many people believe, African countries have not borrowed much from China. However, when considering the ratio of Chinese-debt-to-GDP in \autoref{fig:perc-debt-30}, some African countries appear to be highly indebted to China. Djibouti, for instance, had an alarming ratio of 68.5\% of its GDP consisting of Chinese debt, while Tonga, Niger, and Zambia had ratios exceeding 10\%. This result is in line with the description in \citet{Eom-18}.

A main finding in \citet*{Horn-Reinhart-Trebesch-21} is that China had become the world's largest creditor to developing countries after 2013, surpassing the amount of the World Bank. \autoref{fig:debt-ts} shows the change in the total amount of debt from different main creditors, including China, World Bank,\footnote{Including the International Development Association (IDA) and the International Bank for Reconstruction and Development (IBRD) } IMF, and the aggregation of all countries in the Paris Club. The debt amount started to rise rapidly after 2000, when the China government launched the ``Go Out Policy'' in 1999. In 2017 the debt to China globally had reached \$355 billion, while the debt to the World Bank was \$300 billion.

The main focus of my thesis is to evaluate the debt sustainability for a country after it receives a considerable amount of loans from China. Among all countries, Sri Lanka and Pakistan are often under the discussion regarding the debt-trap issue. From the geostrategic aspect, Sri Lanka could serve as a military base for China \citep{Chellaney_2017}, and Pakistan allows China to better connect with the Arabian Sea \citep{Hurley19-8-debt-trap}. A preliminary view of the change in debt to China for the two countries in Figure \ref{fig: LAK-PAK-debt-ts} also provides some idea of how China is accumulating an increasing amount of debt to the two countries. I briefly introduce the debt-trap diplomacy narrative regarding the two countries in the following sections.

\section*{Sri Lanka}
In the original article where the terminology ``Debt-trap Diplomacy'' was coined, \citeauthor{Chellaney_2017} specifically mentioned the predicament faced by the Sri Lankan government.

\citep*{Moramudali_2020}
\section*{Pakistan}
Similar to Sri Lanka, we observe the abrupt increase on the debt to China, as it already has a relatively high debt to other official creditors. \autoref{fig: pakistan-debt-ts} shows the change in composition of creditors to Pakistan. 
Pakistan is the centerpiece of the China Pakistan Economic Corridor (CPEC), a 3000 km corridor that connects China with the Arabian sea.
CPEC serves as an important network as it reduces the passage for China's energy import from the Middle Eastern countries \citep*{CPEC-wiki}.

China has launched enormous amounts of infrastructure in Pakistan after 2015. These items include a deep water port, road and rail lines, and most importantly, energy sector projects. Up to 2018, the estimation of the total value of projects under the CPEC is \$62 billion, out of which around \$33 billion is allocated for energy projects~\citep*{Hurley19-8-debt-trap}. China is expected to finance about 80 percent of this amount. The private investments for energy projects in Pakistan will be financed by the Exim Bank of China at an interest rate of 5-6\%. 
Private Independent Power Producers (IPP) will be responsible for constructing the energy projects under CPEC, instead of the governments of China or Pakistan. In turn, the government of Pakistan will be legally bound to buy electricity from these companies at rates that were agreed upon before.
However, despite this significant investment, some projects have already been cancelled, such as three major road projects that were cancelled at the end of 2017 \citep*{Hurley19-8-debt-trap}.

In the case of Pakistan, the sudden increase of debt to China draws the attention of researchers and journalists. For example, a report from the Financial Time titled ``Pakistan is on the brink'' states that Pakistan is following Sri Lanka into default. Given the recent frequent analogy drawn between Pakistan and Sri Lanka, it is essential to analyze Pakistan from the perspective of the sovereign default model.


The remaining chapters of the thesis are organized as follows:
In Chapter \ref{ch:lit}, a comprehensive literature review is conducted on the topics of Debt-trap diplomacy and sovereign default models.
Chapter \ref{ch:model} outlines the specific sovereign debt model that will be applied in this study.
Chapter \ref{ch:result} presents the calibration of the model and provides the corresponding empirical results.
Finally, Chapter \ref{ch:conclusion} concludes the thesis.