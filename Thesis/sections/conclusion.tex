%% Recap of the Research Objectives
The objective of this study is to empirically investigate the phenomenon of the debt-trap diplomacy within the economic context. The study employs a visualization approach using an up-to-date sovereign default model to analyze the cases of Sri Lanka and Pakistan, as both countries are strategically important to China.

%% Summary of Findings:
In the empirical analysis carried out in this thesis, I argue that Sri Lanka is classified as belonging to the ``type-one debt trap''. This classification is characterized by a previously stable financial state that is disrupted by external intervention, leading to the likelihood of default. Conversely, Pakistan exhibits characteristics of the ``type-two debt trap'', wherein a financially distressed country that is already in the default crisis becomes even more incapable of repaying its debt, eventually resulting in a heavier default risk.

%% Implications and Significance
The identification of default status and the classification of different types of debt traps through a theoretical framework have significant implications for future research and contribute to a more objective discussion of the debt-trap narrative. The utilization of graphical representation also facilitates a more accessible and comprehensive understanding of the issue. This approach enables scholars to explain the dynamics of debt traps in a visually intuitive manner.

% Limitations
It is important to note that this study has certain limitations. The inspection of a country requires the calibration of the economy, which is oftentimes difficult to conduct as some concepts in the model are ambiguous in reality, such as the probability of reentry to the international credit market after exclusion. The years in default are not completely robust to the filtering method, as a different cyclical component might yield a slightly different debt-output point, as in the case of Pakistan.

%% Suggestions for Future Research
Future research can enhance the ability to capture other important stylized facts by exploring the selection of alternative analytical models. In the context of debt negotiations and partial defaults commonly observed in interactions with China, it is worth considering models that incorporate these dynamics. The current model used in this thesis assumes complete default, necessitating focus on the unsecured portion of debt during the visual presentation.

It is also important to note that the counterfactual comparison, which simply excludes debt from China, does not consider the growth generated by investments under the Belt and Road Initiative. This omission may result in an overestimation of the debt-to-GDP ratio. Future research could explore methodologies that account for the specific impact of BRI investments on GDP growth, allowing for a more accurate assessment of the counterfactual scenario in the absence of Chinese involvement.
One possible approach is to incorporate the business cycle into the model, following \citet{Mendoza-Yue-12}. By including the working capital constraint, the model can better capture the effects of lacking borrowing under international trading.
Researchers in the field of political science can also extend categories of debt traps with quantitative definitions via a similar approach.