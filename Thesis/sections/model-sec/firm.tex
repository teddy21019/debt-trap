Perfectly competitive firms produce nontradable goods $y^N_t$ according to the production technology
\begin{equation}
    y^N_t = F(h_t),
\end{equation}
where $F$ is strictly increasing and strictly concave. Each firm maximizes its profit by choosing the amount of labor. Profit is given by
\begin{equation}
    \Phi_t(h_t) = P^N_t F(h_t) - W_t h_t,
\end{equation}
and the optimal labor demand is then
\begin{equation*}
    P^N_t F'(h_t) = W_t.
\end{equation*}
Dividing both side by the price of tradable goods, and define $w_t \equiv \frac{W_t}{P^T_t}$ as the real wage in terms of tradable goods, the first order condition can be written as
\begin{equation}
    \label{eq:firm-FOC}
    p_t F'(h_t) = w_t.
\end{equation}