The model assumes that the economy is populated by a large number of representative households who maximize their expected lifetime utility
\begin{equation}
    \label{eq:utility}
    E_0 \sum_{t=0}^\infty \beta^t U(c_t),
\end{equation}
where $\beta \in(0,1)$ denotes the discount factor,
and $c_t$ represents the consumption good, which is composed of
tradable consumption $c_t^T$ and nontradable consumption $c_t^N$.
Assume that $c_t$ follows an aggregate technology
\begin{equation}
    \label{eq:A}
    c_t = A(c^T_t, c^N_t),
\end{equation}
where $A$ is an increasing, concave, and linearly homogeneous function that captures characteristics such as the ratio or elasticity of substitution between tradable and nontradable consumption.
The period utility function $U(c_t)$ follows the standard assumption, which is a strictly increasing and strictly concave function.

Assume that households only have access to the one-period and state non-contingent bond.
The households spend on consumption of tradable and untradable goods, along with their debt which is realized at this period. Their resources consist of labor incomes, dividend incomes, lump-sum transfers, as well as debt incomes. The households are also endowed with tradable goods, which follow a stochastic process.
The budget constraint of the representative household is then
\begin{equation}
    \label{eq:bc}
    P^T_t c^T_t + P^N_t c^N_t + P^T_t d_t =
    P^T_t \tilde{y}^T_t + W_t h_t + (1- \tau^d_t)P^T_t q^d_t d_{t+1} + F_t + \Phi_t,
\end{equation}
where $P^T_t (P^N_t)$ denotes the nominal price of tradable (nontradable) goods, $d_t$ the bond denominated in tradable goods which is due in period $t$, $q_t$ the price of debt to be repaid at $t+1$, $\tilde{y}^T_t$ the endowment of traded goods to the household, $W_t$ the nominal wage, $h_t$ the hours worked, $\tau^d_t$ the tax on debt, $F_t$ a lump-sum transfer from the government, and finally $\Phi_t$ the nominal profits from owning firms.
Households' working hour is bounded by an upper limit
\begin{equation}
    \label{eq:h-constraint}
    h_t \le \bar{h},
\end{equation}
and they take the working hours $h_t$ as given.

The households' problem is to choose $\{c_t, c_t^T, c_t^N, d_{t+1}\}$ such that their utility \eqref{eq:utility} is maximized subjected to the budget constraints \eqref{eq:A} -- \eqref{eq:h-constraint} and the no-Ponzi-game debt limit.
Further denote the relative price of nontradable in terms of tradable goods as $p_t \equiv \frac{P^N_t}{P^T_t}$, we have the following first order conditions
\begin{subequations}
    \begin{align}
        p_t &= \frac{A_2(c_t^T, c_t^N)}{A_1(c_t^t, c_t^N)}\\
        \lambda_t &= U'(c_t)A_1(c_t^T, c_t^N)\\
        (1-\tau_t^d)q_t^d \lambda_t &= \beta E_t \lambda_{t+1},
    \end{align}
\end{subequations}
where $\lambda_t$ is the Lagrange multiplier.