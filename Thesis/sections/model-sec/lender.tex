The behavior of foreign lenders is not explicitly modeled in this framework, but as all are rational agents, the expected marginal benefit of lending to the domestic country must be equivalent to the opportunity cost of funds.
Let $r^*$ represent the opportunity cost for the foreign lenders, which could be the world interest rate. Since $q_t$ is the price of debt that repays one unit of $d_{t+1}$ tomorrow, the return on the debt is $\frac{1}{q_t}$. The lenders take the risk of default into consideration, and therefore the expected return will actually be lower. Assume that foreign lenders are risk neutral and do not require a risk premium, prompting:
\begin{equation}
    \label{eq:lender}
    \frac{\Pr(I_{t+1}=1 \mid I_{t}=1)}{q_t} = 1 + r^* .
\end{equation}
Equivalently, the equation can be written as:
\begin{equation*}
    I_t \left[ q_t - \frac{E_t I_{t+1}}{1+r^*} \right] = 0.
\end{equation*}
