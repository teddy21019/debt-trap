I assume here, under the lack of enforcement in the international credit market that the government has the option to benevolently free up its domestic balance sheet by choosing to default or not.
Denote $I_t$ as the indicator of whether the government chooses to honor its debt in period $t$. If the government repays in this period ($I_{t} = 1$), then the country will be able to borrow in the following period, and hence $d_{t+1} > 0$. However, if the government chooses to default ($I_t = 0$), then the country will enter the status of financial autarky and is unable to have any sovereign debt in the next period, hence $d_{t+1} = 0$. The above scenario can be written as a slackness condition:
\begin{equation}
    \label{eq:gov-next-debt}
    (1 - I_t)d_{t+1} = 0 .
\end{equation}

To model the duration of financial exclusion, assume that once the country is in bad standing in the international credit market, it can regain a fiscally sound reputation and access to financial markets with probability $\theta \in [0,1)$, and remain in bad standing with probability $1-\theta$. This implies that the country has an average exclusion duration of $\frac{1}{\theta}$ periods\footnote{
    The expected exclusion period $= \sum_{t=1}^{\infty} t \theta (1-\theta)^{t-1} = \theta  \sum_{t=1}^{\infty} t (1-\theta)^{t-1} = \frac{1}{\theta}$.
}

Assume that the government distributes the proceeds from the debt tax to households as a lump-sum payment. If the government honors the debt, then it repays $d_t$, but if the government decides to default, then it will not make any payments to foreign lenders, and instead will return any payments made by households directly to them. The budget constraint for the government can then be expressed as:
\begin{equation}
    \label{eq:gov-budget}
    f_t = \tau_t^d q_t^d d_{t+1} + (1-I_t)d_t,
\end{equation}
where $f_t \equiv \frac{F_t}{P^T_t}$ is the lump-sum transfer in terms of tradable goods. The right-hand side of the equation states that the transfer to households will include $d_t$ when $I_t = 0$, which is when the country decides to default. Nevertheless, the transfer of debt tax will be zero after default since $d_{t+1} = 0$ when $I_t = 1$, according to Equation \eqref{eq:gov-next-debt}.