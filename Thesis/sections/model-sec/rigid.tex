The key assumption in \citet*{Schmitt-Uribe-16} and \citet*{Na-18} is the downward nominal wage rigidity.
As the wage is unable to be adjusted to a lower level, involuntary unemployment is inevitable, hence the government has the incentive to allow devaluation. The model imposes a lower bound to the growth rate of nominal wage
\begin{equation}
    W_t \ge \gamma W_{t-1}, \qquad \gamma > 0.
\end{equation}
This implies that the growth rate $\frac{W_{t} - W_{t-1}}{W_{t-1}} \ge \gamma - 1$. When this inequality is unbinding ($W_t > \gamma W_{t-1}$), the economy is fully employed ($h_t = \bar{h}$). However, if the condition binds, the economy might have unemployment ($h_t < \bar{h}$). This relationship can be written as the following equation
\begin{equation}
    \label{eq:wage-rigid}
    (\bar{h} - h_t)(W_t - \gamma W_{t-1}) = 0.
\end{equation}
