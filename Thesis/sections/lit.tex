\section{Debt-Trap Diplomacy}
Whether the debt-trap diplomacy is just a conspiracy used as an instrument for western countries to justify their political strategy or is in fact causing stress on the receiving counties intentionally has continuously seen great debates and arguments.

%% +1
The term ``debt-trap diplomacy'' was first coined by \citet{Chellaney_2017}, who states that the infrastructures supported financially by the China government in Sri Lanka are burdensome and causing Sri Lanka, as well as other small and poor countries, to endure the unsustainable loans, forcing them to cede strategic leverage to China.
%% +2, +4, +5
From a political aspect,
researchers from the Belfer Center for Science and International Affairs propose that China may pursue three main strategic objectives using this approach \citep*{Parker2018}.
These objectives include:
\begin{enumerate*}[label = (\roman*)]
    \item expanding its ``String of Pearls'' to address the ``Malacca Dilemma'' and extend its influence along crucial South Asian trade routes,%
    \footnote{
        The Strait of Malacca, located between Malaysia and Indonesia, is one of the busiest and most critical shipping routes in the world, connecting the Indian Ocean and the Pacific Ocean.
        The ``Malacca Dilemma'' refers to the strategic vulnerability faced by China due to its heavy dependence on the Strait of Malacca for maritime trade and energy imports \citep*{Parker2018}.
    }
    \item destabilizing and fragmenting the regional coalition led by the United States that challenges China's claims in the South China Sea, and
    \item facilitating the People's Liberation Army Navy (PLAN) in advancing beyond the "Second Island Chain" and into the open waters of the Pacific Ocean.%
    \footnote{
        The Second Island Chain encompasses a collection of islands, such as the Bonin Islands and Volcano Islands of Japan, the Mariana Islands (notably Guam), the western Caroline Islands, and extending towards Western New Guinea. It serves as the delineation of the eastern maritime boundary of the Philippine Sea, while also incorporating Guam, an American overseas territory housing a strategically fortified military base
        \citep{Vorndick-Island-chian-18}.
    }
\end{enumerate*}
% +3
This is considered as the reshaping of ``soft'' infrastructure that China is hoping to enhance \citep{Jonathan-Hillman-18}.

% +?
A common criticism concerning the loans from China is that the terms and conditions are typically not transparent compared to other creditors in order to encourage those debtor countries' dependency to China \citep{tillerson2018us}. For example,
an analysis on the contract of China's long foreign lending suggests that the terms of these agreements feature unusual confidentiality clauses, seek advantages over other creditors through collateral arrangements, and potentially grant influence over debtors' policies \citep{Gelpern-22}.
The authors reveal that the contracts signed with Chinese state-owned entities after 2014 frequently contain extensive clauses to maintain confidentiality that impose broad obligations on the debtor to keep contract details and related information undisclosed. Furthermore, about 30\% of these agreements require the borrowing country to establish a dedicated bank account, typically controlled by the lender, as a form of collateral for debt repayment. These revenue accounts, uncommon in sovereign lending, restrict the borrowing country's authority over its own finances. Additionally, compared to other types of debt contracts, China's loans agreements more commonly include cross-default clauses, allowing lenders to demand immediate repayment (called an \emph{acceleration}) if the borrower defaults on other lenders.%
\footnote{
    These agreements allow China to jump the ``seniority queue'', and hence making debt to China de facto seniority debts \citep{Chen-Muyang-23-china-debt-seniority}.
}
These contract characteristics, along with their creative design to manage credit risks and enforcement hurdles, portray China as a powerful and commercially astute lender to developing nations.

There are several studies regarding the estimation of debt vulnerability caused by China's loans. \citet*{Hurley19-8-debt-trap} evaluate the debt sustainability in BRI countries by examining their dept-to-GDP ratio versus their share of China's debt. Following the threshold of 50-60\% rising debt-to-GDP ratio constructed by \citet{Chudik-15}, they identify eight
countries that are particularly risky.\footnote{
    These countries are Djibouti, Kyrgyzstan, Laos, the Maldives, Mongolia, Montenegro, Pakistan, and Tajikistan.}
\citet*{Bandiera-Vasileios-BRI-debt} examine the impact of investment and infrastructure projects under BRI on the debt-to-GDP ratios of recipient countries. The authors analyze the growth effects of BRI investment and estimates the potential increase in debt vulnerabilities for certain countries through a model-based growth projection. The findings suggest that approximately 28\% of BRI investment recipient countries, consisting of 7 low-income developing countries and 5 emerging markets, are expected to face increased debt vulnerability in the medium term, while 37\% of countries, including 5 low-income developing countries and 6 emerging markets, may experience a rise in their debt-to-GDP ratio due to BRI investment and financing in the long term, with 8 of them being vulnerable to changes in financing costs.


In stark contrast, critics of the ``debt-trap diplomacy'' narrative often argue that the benefit of China's lending on the receiving country is neglected, and state that the concerns are often exaggerated.
% -3
\citet*{Eom-18} argue that Chinese loans did not play a significant role in causing debt distress in Africa. They identify 17 African low-income  countries that were and under high risk of debt distress and find for less than half (8) of them, their levels of debt to China were relatively small compared to their total external debt, and that the debt distress was caused primarily by other conflicts in the nation; six other countries had higher loans from China, but also to other official creditors; and in only three countries, Chinese loans were the main contributions to the debt distress.
% -4

\citet*{Brautigam-meme-2020} indicates that debtor countries have voluntarily accepted Chinese loans and report positive experiences, suggesting that concerns over Chinese infrastructure funding are exaggerated, as many view China as an appealing economic model and development partner.
In the case of Sri Lanka, \citet*{Brautigam-meme-2020} also argues that the project of Hambantota Port was the concept of former President Mattala Rajapaksa, and that Chinese banks have shown willingness to assist their restructuring of existing loans.
% -5
The Rhodium Group presents 40 cases of China's external debt renegotiations, and finds that not only are asset seizures a rare occurrence, but China is limited in its leverage during negotiation due to external events such as change in leadership \citep*{Rhodium-DTD-19}.

Governments of indebted countries often defend their own loan decisions. The Minister of Finance of the Trinidad and Tobago, for instance, argued that choosing a loan without the need for retrenchment, currency devaluation, or other adverse measures, especially when the interest rates are similar, is an obvious and favorable choice.\footnote{Loop News ``Imbert: Choosing between IMF, Chinese loan a `no-brainer,'''June 15, 2021.}

\section{Sovereign Debt Model}
Sovereign debt models under the Eaton-Gersovitz framework have been widely used to analyze default decisions driven by reputation and sanction \citep*{Eaton-Gersovitz-81}.
Several important features are included in the framework to replicate important stylized facts not captured under the benchmark model of Eaton-Gersovitz.
For instance,
\citet{Aguiar-Gopinath-06} and \citet{Arellano-08} take into consideration the risk premium in the market price of debt. As a result, country spread, defined by the difference between the country's net interest rate and the world risk-free interest rate, can be accounted for quantitatively.
\citet{Na-18} further investigate the role of government's optimal policy under wage rigidity in a decentralized economy to study the \emph{Twin Ds} phenomenon. They find that being able to freely set the optimal taxation rate and devaluation rate improves the welfare of a country by reducing unemployment, which provides further incentive for a country to default.

\citet*{Mendoza-Yue-12} endogenize output cost by combining sovereign default and business cycle, thus offering a microfoundation for the ad-hoc assumption of output costs during default episodes in \citet{Arellano-08}. The firm in the economy faces a working capital constraint that amplifies the effect caused by a shock in total factor productivity.
\citet*{Chatterjee-12} consider the fact that sovereign bonds have long and short maturities, and the authors extend the model such that their maturity is realized with a probability.

Seniority of debt is another popular topic in recent literature. Seniority is the priority of a debt contract among all debts. An empirical study by \citet*{Schlegl-Trebesch-Wright-19} shows that private creditors are often paid first, compared to bilateral official creditors, and multilateral institutions (IMF and World Bank) are indeed senior creditors. \citet*{Chatterjee-15-seniority} propose a model to analytically track the seniority of debts. \citet*{Ho-Ritschl-23} extend the Eaton-Gersovitz model to include both senior and junior debts, representing either commercial debts or war reparations. The assignment of seniority for the different debts can then be switched, which the authors refer to as ``seniority reversal.''

Another important stylized fact regarding sovereign defaults is that these episodes are in general partial defaults, contrary to the assumption that a country resets its debt levels after a temporary exclusion from the international financial market \citep[see][]{Sturzenegger-Zettelmeyer-08,Cruces-Trebesch-13}. The fraction that a country chooses to default is called \emph{haircuts} and is usually determined through debt renegotiations.
\citet{Yue-10} combines the benchmark Eaton-Gersovitz model with a Nash bargaining game model to simultaneously explain default and debt negotiations. The author proves that the bargaining power has a great impact on debt recovery rate and bond spread.
\citet{Bi-08} incorporates the Eaton-Gersovitz model with a stochastic bargaining model to explain why delays in debt negotiation can actually be beneficial.
However, empirical data show that debt does not necessarily drop following a restructuring; debt during the episode in fact increases first, creating a hump-shaped dynamic. \citet{Arellano-23-partial-default} emphasize this stylized fact and propose a model and accounting framework that explains the phenomenon. \citet{sovereign-debt-review-yue-19} provide a comprehensive review of the literature on sovereign debt.


A rich body of literature examines default episodes using calibration on Argentina \citep{Arellano-08, Schmitt-Uribe-16,Mendoza-Yue-12,Na-18}. \citet*{Hinrichsen_2020-chapter4} examines the effect of war reparations on countries' default set using data from France in the 1870s, Germany in the 1930s, and Finland in the 1940s. \citet*{Ho-Ritschl-23} investigate transfer protection in the Dawes Plan by calibrating the German economy during the 1920s.