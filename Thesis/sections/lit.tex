\section*{Debt-Trap Diplomacy}
Whether the debt-trap diplomacy is just a conspiracy used as an instrument for the western countries to justify its political strategy, or is it in fact causing stress on the receiving counties intentionally, is continuously under great debates and defenses.

%% +1
\citet{Chellaney_2017} first stated that the infrastructures supported financially by the China government in Sri Lanka is burdensome and causing small and poor countries to endure the unsustainable loans, forcing them to cede strategic leverage to China.
%% +2, +4, +5
From a political aspect,
researchers from the Belfer Center for Science and International Affairs propose that China may pursue three main strategic objectives using this approach \citep*{Parker2018}.
These objectives include: expanding its ``String of Pearls'' to address the ``Malacca Dilemma\footnotemark{}'' and extend its influence along crucial South Asian trade routes, destabilizing and fragmenting the regional coalition led by the United States that challenges China's claims in the South China Sea, and facilitating the People's Liberation Army Navy (PLAN) in advancing beyond the "Second Island Chain" and into the open waters of the Pacific Ocean.
\footnotetext{The Strait of Malacca, located between Malaysia and Indonesia, is one of the busiest and most critical shipping routes in the world, connecting the Indian Ocean and the Pacific Ocean.
The ``Malacca dilemma'' refers to the strategic vulnerability faced by China due to its heavy dependence on the Strait of Malacca for maritime trade and energy imports \citep*{Parker2018}.}
% +3
This is considered as the reshaping of ``soft'' infrastructure that China is hoping to enhance \citep{Jonathan-Hillman-18}.
% +?
An analysis on the contract of 100 China's long foreign lending suggests that the terms of these agreements feature unusual confidentiality clauses, seek advantages over other creditors through collateral arrangements, and potentially grant influence over debtors' policies \citep{Gelpern-22}. These contract characteristics, along with their creative design to manage credit risks and enforcement hurdles, portray China as a powerful and commercially astute lender to developing nations.

There are several studies regarding the estimation of debt vulnerability on caused by China's loan. \citet*{Hurley19-8-debt-trap} studies the debt sustainability in BRI countries by examining their dept-to-GDP ratio versus the share of China's debt. Following the threshold of 50-60 percent rising debt-to-GDP ratio constructed by \citet{Chudik-15}, they identify eight
countries\footnote{
    These countries are Djibouti, Kyrgyzstan, Laos, the Maldives, Mongolia, Montenegro, Pakistan, and Tajikistan}
that are particularly risky.
\citet*{Bandiera-Vasileios-BRI-debt} examines the impact of investment and infrastructure projects under the BRI on the debt-to-GDP ratios of recipient countries. It analyzes the growth effects of BRI investment and estimates the potential increase in debt vulnerabilities for certain countries through a model-based growth projection. The findings suggest that approximately 28\% of BRI investment recipient countries, consisting of 7 low-income developing countries and 5 emerging markets, are expected to face increased debt vulnerability in the medium term, while 37\% of countries, including 5 low-income developing countries and 6 emerging markets, may experience a rise in their debt-to-GDP ratio due to BRI investment and financing in the long term, with 8 of them being vulnerable to changes in financing costs.


In stark contract, critics of the debt-trap diplomacy narrative often argue the benefit of China's lending on the receiving country, and state that the concerns are often exaggerated.
% -3
\citet*{Eom-18} argues at the time (2018) that Chinese loans did not play a significant role in causing debt distress in Africa.
% -4
\citet*{Brautigam-meme-2020} indicates that debtor countries have voluntarily accepted Chinese loans and report positive experiences, suggesting that concerns over Chinese infrastructure funding are exaggerated, as many view China as an appealing economic model and development partner.
In the case of Sri Lanka, \citet*{Brautigam-meme-2020} also argues that the project of Hambantota Port was the concept of former President Mattala Rajapaksa, and that Chinese Banks have shown willingness to assist their restructuring of existing loans.
% -5
The Rhodium Group reviews 40 cases of China's external debt renegotiations, and finds that not only are asset seizures a rare occurrence, but China is limited in the leverage in negotiation due to external events such as change in leadership \citep*{Rhodium-DTD-19}.

Notably, governments of the indebted countries often defends their own decision of loan. The Minister of Finance of the Trinidad and Tobago, for instance, argued that choosing a loan without the need for retrenchment, currency devaluation, or other adverse measures, especially when the interest rates are similar, is an obvious and favorable choice\footnote{Loop News ``Imbert: Choosing between IMF, Chinese loan a `no-brainer,'''June 15, 2021}. The former President of Sri Lanka also defended that the Hambantota Port is not a debt-trap.

\section*{Sovereign Debt Model}
Sovereign debt models under the Eaton-Gersovitz framework has been widely used to analyze default \citep*{Eaton-Gersovitz-81}.
Developed by \citet{Aguiar-Gopinath-06} and \citet{Arellano-08},
several important features are included in the framework recently.
For instance,
\citet{Schmitt-Uribe-16} extends the model by implementing the nominal wage rigidity into the model, and \citet{Na-18} further investigates the role of government's optimal policy under wage rigidity in a decentralized economy to study the Twin Ds phenomenon.
In the output models mentioned above, output cost is set exogenously with an ad hoc loss function; \citet*{Mendoza-Yue-12} endogenizes output cost by combining sovereign default and business cycle.
A rich literature examines the default episodes using calibration on Argentina \citep{Arellano-08, Schmitt-Uribe-16,Mendoza-Yue-12,Na-18}. \citet*{Hinrichsen_2020-chapter4} examines the effect of war reparations on countries' default set using data from France in the 1870s, Germany in the 1930s, and Finland in the 1940s. \citet*{Ho-Ritschl-23} investigate transfer protection in the Dawes Plan by calibrating the German economy during the 1920s.