As described in Section \ref{sec: model-default-decision} and Appendix \ref{ap: default-set}, the default decision is jointly decided by the current debt level and its output for tradable goods during the period. Under a certain debt stock $d_t$, the less the tradable output $y^T_t$ it is endowed with, the lower the benefit a country receives when it chooses to repay its debt.
Ultimately, if the output falls below a certain threshold value, then it becomes optimal for the country to default.
\citet{Hinrichsen_2020-chapter4} compares the theoretical scenarios of defaulting with the actual data through a graphical approach. He maps the default set evaluated via value function iteration into the state space $(y^T_t, d_t)$, with the output as the horizontal axis and the debt level as the vertical axis. A grid on the space is colored in gray if the corresponding debt-output pair is \emph{not} in the default set.

Figure~\ref{fig:ds} demonstrates the default set for Sri Lanka and Pakistan. The debt level in the default set plot varies between the two countries due to the difference in the calibration and the targeted equilibrium moments. As shown in the figure and proven in Appendix \ref{ap: default-set}, the upper limit of a default set (the intersection of the white and gray areas given a certain debt level) increases as the debt level increases, indicating that it requires a higher endowment of tradable output in order to incentivize the country to continue to fulfill its obligations to repay.\footnote{
    In the default set depicted in \citet{Hinrichsen_2020-chapter4}, there is a discontinuity in the upper bound of the default set. This indicates that when the output level drops to a certain threshold, the country chooses to default, regardless of the current debt level. This is contradictory to the analytical conclusion proved in Proposition \ref{prop3} in Appendix \ref{ap: default-set}. During private correspondence, the author acknowledges this mistake in his doctoral thesis, but he has already corrected it in his latest book.
}

The actual data are then plotted on the default set for comparison. The output level corresponds to the cyclical component of the per capita tradable-GDP obtained by the HP-filter for the respective year.
The corresponding debt level is calculated by multiplying the debt-to-nominal-tradable-GDP ratio by the haircut ratio and 4 quarters. This calculation is necessary, because the targeted debt level during calibration specifically refers to the quarterly unsecured portion of the debt, and thus the data should undergo the same calculation to accurately represent the unsecured debt component.

The debt data used in this analysis consists of two components: the debt stock excluding China obtained from the International Debt Statistics (IDS), and the debt stock specifically for China sourced from the database created by \citet*{Horn-Reinhart-Trebesch-21}. It is worth noting that China government loans channeled through state-owned enterprises are typically not reported in the World Bank Debt Reporting System (DRS), which is the basis for IDS. This omission leads to an underestimation of the actual debt burden faced by the country. The additional loans provided by state-owned commercial banks, referred to as ``hidden debts'' by \citet*{Horn-Reinhart-Trebesch-21}, are included in their database. In order to capture the total debt burden accurately, the total debt stock reported in IDS is adjusted by excluding the debt to China from IDS and adding the debt to China obtained from the database by \citet*{Horn-Reinhart-Trebesch-21}.

\subsection{Default Decision for Sri Lanka}

Figure \ref{fig: ds-sri-data-with-china} plots the debt-output pair for Sri Lanka from 2007 to 2017. Each point represents the actual data pair for the year, adjusted by the haircut ratio. Recall that China initiated its constructions in infrastructures such as the Hambantota International Port in 2007, and Mattala Rajapaksa International Airport in 2009. The selection of this time span allows us to examine the change in debt burden for Sri Lanka comprehensively in a chronological way.

As shown in the figure, at the initial phase of China's investment in 2007 and 2008, Sri Lanka is still under its safe zone of not defaulting, despite that it experienced a little drop in the output. Sri Lanka first landed into the default set during 2009, as debt to China reached \$3 billion. During 2010 and 2011, Sri Lanka ended its 25-years civil war, and experienced an unprecedented output growth. The agriculture sector in 2010 grew 7\%, and the industrial sector grew 8\%. This rapid increase in GDP largely lowered the debt-to-tradable-GDP ratio, both in 2010 and 2011, and hence Sri Lanka was able to avoid the risk of default and stayed under the non-default set. However, as GDP growth in tradable goods slowed down to a rate of 5\%, debt to China increased by 6\%, which slowly pushed Sri Lanka to the brink of default. Eventually, in 2013, as total debt reached \$41 billion, in which China accounted for 13\%, Sri Lanka was again entering the default set. The status continued until 2016, with 2014 being an ambiguous exception.

It is worth mentioning in 2015 that as President Maithripala Sirisena defeated former President Mahinda Rajapaksa in the election, he promised to extricate Sri Lanka from the debt burden from China. However, according to Figure \ref{fig: ds-sri-data-with-china}, Sri Lanka was too indebted during 2015 and 2016, and it would be difficult for the new government to take actions besides undergoing debt restructuring. Eventually, in 2017, the infamous event of Sri Lanka leasing the Hambantota International Port for 99 years and selling 70\% of the stake to China Merchants Port stroke the headline. This action, however, did not write off the loans for the construction of the infrastructures, as it was being used as a raise in money to repay debt to other official creditors \citep{Brautigam-meme-2020, Moramudali_2019}. Regardless of the narrative of whether the lease is involuntary or not, along with the outstanding output level in 2017, Sri Lanka was back to the non-default set after the lease.
All the cyclical components of the tradable goods (the x coordinate) are above 1 after 2011. This indicates that Sri Lanka is under the default set even when output is mostly above the steady state level of tradable outputs.

Sri Lanka did not have issues on defaulting before 2007. Debt to China indeed boosted Sri Lanka into the predicament of debt unsustainability, or in other words, pushed Sri Lanka into a debt trap. I refer to this as the ``type-one debt trap.'' Formally defining this, the ``type-one debt trap'' refers to a scenario where a financially stable country is pushed into a state of debt unsustainability, leading to a high risk of default on its debt obligations. This debt trap occurs as a result of external interventions, particularly an influx of excessive loans, which disrupt the financial equilibrium of the country.

An intriguing question arises: What if the debt to China is not included, holding other conditions fixed? \citet{Hinrichsen_2020-chapter4} compares the debt for countries in war reparations with and without indemnities, and here I adopt the same approach. Figure \ref{fig: ds-sri-data-x-china} shows the comparison of the data. In the figure, the solid black dots represent datapoints including debts owed to China, while the gray dots represent datapoints excluding debts from China.
It is obvious that, \emph{ceteris paribus}, all years spanning from 2009 to 2017 are not under the default set.

This above is of course an inexact comparison.
First, the debt level given in the model is endogenous, and it is incapable of distinguishing between China and other lenders. It is possible that Sri Lanka sought other creditors for loans if the infrastructures were planned voluntarily, as mentioned in \citet{Brautigam-meme-2020}. In this case the debt level in the absence of China can be underestimated.
Second, the impact on GDP would be uncertain if China had not been involved in the construction of the infrastructure projects. For instance, \citet{Bandiera-Vasileios-BRI-debt} estimate from a long-term growth model in 2020 that the additional growth from investment on the Belt and Road Initiative (BRI) is about 0.08\%.
Estimations of the extra growth due to investments during 2009 to 2017 are crucial in order to examine the counterfactual debt burden in the absence of China's intervention. If investments in China indeed contributed to GDP growth during the periods, then the debt-to-tradable-GDP ratio excluding China is also underestimated.

\subsection{Default Decision for Pakistan}

Figure \ref{fig: ds-pak-data-with-china} presents the debt-output pair for Pakistan spanning from 2010 to 2017. The initiation of extensive infrastructure projects under the China-Pakistan Economic Corridor (CPEC) took place in 2015, marking a significant milestone. While the focus of comparison primarily revolves around the two years preceding China's intervention, including the data from 2010 allows for a comprehensive understanding of Pakistan's debt situation.

Pakistan encountered the most severe flooding in 2010, which caused an estimated \$9.7 billion loss in infrastructure. The devastation brought Pakistan to the default set in 2010.
However, since the disasters in 2010 and 2011, Pakistan was not under the default set during 2013. It is relatively safe from being in the default status, as even an 8\% contraction in output (if $y^T_t$ = 0.92) does not push Pakistan to default.

The debt level started to increase intensively, and up to 2015, debt to China had hit about \$15 billion, surpassing the amount of debt to the World Bank, which at that time was about \$13 billion. Since then, China has become the largest creditor to Pakistan (see Figure~\ref{fig: pakistan-debt-ts}). As a result, debt-to-tradable GDP increased, and both datapoints in 2014 and 2015 are located above 2013 in Figure~\ref{fig: pak_d_t_trad}. In 2016, 6 additional major loans on infrastructure constructions mainly on the energy sector were launched, which summed up to \$8.6 billion, and hence in 2016 Pakistan was around the brink of default. Details about the projects are described in Section \ref{sec: intro-pak}.

Eventually in 2017, with 3 more major loans in power projects initialized, the total debt stock when up to \$108 billion, and compared with nominal tradable GDP of \$136 billion, the debt-to-tradable-GDP is 80\%, which corresponds to about 120\% unsecured debt per quarter. This pushed Pakistan into the default set, and made Pakistan strangled in the debt trap.

The provision of a large number of power projects to Pakistan by China is primarily aimed at addressing its most urgent needs in power supply. Pakistan has been suffering from energy shortage since the 1990s. During 2010 to 2013, the former Pakistan Prime Minister Yousuf Raza Gilani launched an energy policy that aimed to resolve the power crisis, with measures including banning neon signs, closing street markets early, and introducing 13 independent power producers (IPPs).\footnote{
    BBC ``Pakistan's PM announces energy policy to tackle crisis,'' April 22, 2010.
}
While the construction of large-scale power plants, including nuclear plants, could have directly increased energy supply, the Pakistani government primarily focused on energy conservation policies. Financial restrictions may have been the main concern for the government that prevented it from initializing large constructions, especially after the tragic natural disasters such as the 2010 floods in Pakistan.\footnote{
    Asian Development Bank and World Bank jointly estimate the total cost of the damage at approximately \$10.1 billion and the reconstruction costs at least \$6.8 billion \citep{pakistan-flood}.
}

China grasped this opportunity and provided excessive aid when Pakistan was under this turmoil, thereby further exacerbating the already onerous debt obligation. I refer to this as the ``type-two debt trap.''
This type of debt trap arises when a financially-distressed country is already insolvent or facing severe financial difficulties, making it challenging to secure additional funds from other countries or lenders.\footnote{
    Moody's rated Pakistan as Caa2 in 2012. Data from Trading Economics.
}
However, a specific entity or source steps in and provides substantial loans to the distressed country, exceeding what would typically be considered reasonable or sustainable levels. Despite the overwhelming nature of these loans, the distressed country, in its desperate need for financing and infrastructures, accepts them. This sets the stage for a desperation-driven debt trap, wherein the distressed country becomes increasingly reliant on the lending entity for funds, exacerbating its financial vulnerability and restricting its options for recovery.

Similar to Sri Lanka, the following attempts to illustrate the economy of Pakistan excluding China. Figure~\ref{fig: ds-pak-data-x-china} illustrates the debt-output relationship when excluding China's interventions. Throughout the entire time span, the debt-to-tradable-GDP ratio remains well below the default threshold. An exception is the year 2010, when the flooding hit Pakistan, as it would default regradless of the debt to China. Notably, in 2017 the ratio closely resembles that of 2013. However, it is important to reiterate that this result does not consider the potential increase in debt owed to other creditors or the benefits derived from investments related to BRI.

As mentioned previously, Pakistan is targeting solutions for its power supply, but is not under the financial status of realizing them. It is reasonable to argue that without the loans and projects initiated by China, Pakistan would not have undertaken these energy projects, and its debt would not have accumulated dramatically from other official creditors.

Another aspect of uncertainty when assessing the counterfactual debt-to-tradable-GDP ratio is the impact on GDP following China's investment in infrastructure. According to estimates from \citet*{Bandiera-Vasileios-BRI-debt}, using a long-term growth model, additional investment from BRI accounted for 1.93\% of GDP, resulting in a 0.41\% increase in GDP compared to the baseline model in 2020. This effect is five times larger than that observed in Sri Lanka. Although estimations for the period between 2013 and 2017 have not been conducted, a preliminary observation can be made from Figure \ref{fig: pak_d_t_trad}. Nominal tradable GDP demonstrates increasing growth after 2013, despite the total debt level increasing at a faster pace. Since this observation is not sufficient for causal inference, further examination of the infrastructure's contribution to GDP is a topic worthy of future research.