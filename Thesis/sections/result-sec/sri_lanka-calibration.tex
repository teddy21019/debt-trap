The model is calibrated to Sri Lanka from 2007 to 2017, when the Chinese government started to provide the increasing amount of loans (see \autoref{fig: sri-lanka-debt-ts}).
The output process of \refeq{eq:ar1-output} is proxied by the measure of tradable GDP. I follow \citet{Schmitt-Uribe-16} in obtaining tradable outputs as the sum of GDP in agriculture, forestry, fishing, mining and manufacturing. Considering that the seasonality in the quarterly data for Sri Lanka might impose a higher volatility estimated in the AR(1) process, I follow \citet{Hinrichsen_2020-chapter4} and estimate the annual data over 1980 -- 2021. I obtain the cyclical component of the tradable outputs by filtering this time series with an HP-filter with the smoothing parameter set to $100$.
Estimation of the AR(1) on the cyclical component thus yields $\rho = (0.93)$ and $\sigma_u = (0.037)$.

The global risk-free world interest rate $r^*$ is set to match the 3-month treasury bill rate during the period. The 3-month treasury bill experienced a continuous low rate since 2008, with a median of $0.12\%$ annually. This gives a quarterly risk-free rate of $0.03\%$.
The probability of reentry is difficult to assess since Sri Lanka encountered its first default in April 12, 2022, and is not yet undergoing the process of restructuring. As a result, following \citet*{Chatterjee-12} and \citet*{Hinrichsen_2020-chapter4}, I set the probability of reentry to $0.0385$, which implies that the country will be in default on average for about $6.5$ years, matching the median of default spans for past 100 systemic crises \citep*{Reinhart-Rogoff-2014-100-episode}.

Calibration on other structural parameter regarding the Sri Lanka economy follows \citet*{Jegajeevan-Sri-Lanka-DSGE}. The author estimates the Sri Lanka economy with a DSGE model. Labor share