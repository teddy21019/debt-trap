The approximated equilibrium is obtained by conducting value function iteration (VFI) over an $n_y \times n_d$ discretized and equally spaced state space,
where $n_y = $ 200 is the number of grids for the output process and $n_d=$ 200 is the number of grids for the debt \citep{Na-18}. Denote $[\underline{y}^T, \overline{y}^T]$ as the lower and upper bound of output grid. Following \citet{Uribe-Schmitt-Grohe-textbook}, this is set as $[-4.2 \sigma_u, 4.2 \sigma_u]$. Also following the authors, since the average debt levels for both countries do not exceed 150\%, the upper bound for debt is set as 1.5, therefore the debt range for the VFI is $[\underline{d}, \overline{d}]=[0,1.5]$.

\citet{Schmitt-Uribe-16} and \citet{Na-18} deal with this issue by constructing a transition probability matrix over the grids of the AR(1) output process.
A time series of 10 million observations was generated based on \refeq{eq:ar1-output}. Each observation is then assigned to the nearest grid point among the 200 discrete values of $\ln y^T$. The discretized series is analyzed to calculate the probabilities of transitioning from one discrete state to another in consecutive periods.
To obtain the transition probability matrix, a $200\times200$ matrix is initialized with zeros. For each pair of consecutive observations, the corresponding element in the matrix is incremented by 1. After considering all the observations, the matrix is normalized by dividing each row by the sum of its elements. This results in the estimated transition probability matrix, which effectively captures the covariance matrices of order 0 and 1
\citep*{Uribe-Schmitt-Grohe-textbook}.

Following \citet{Schmitt-Uribe-16}
and \citet{Na-18}, a simulation of the model based on the policy function is conduct 1.1 million time. After discarding the first 0.1 million periods, the periods in which a default occurs are identified, and a window of 12 quarters prior to and 12 quarters after each default episode is extracted. The median is then computed period by period across all windows, and the period of default is normalized to 0. In this thesis, the simulation of equilibrium dynamics is mainly used in the calibration for $(\beta, \delta_1, \delta_2)$ since they target the equilibrium outcomes of debt-level and the frequency of default.