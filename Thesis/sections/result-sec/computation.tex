The approximated equilibrium is obtained by conducting value function iteration over an $n_y \times n_d$ discretized and equally spaced state space,
where $n_y = $ 200 is the number of grids for the output process and $n_d=$ 200 is the number of grids for the debt \citep{Na-18}. Denote $[\underline{y}^T, \overline{y}^T]$ as the lower and upper bound of output grid. Following \citet{Uribe-Schmitt-Grohe-textbook}, this is set as $[-4.2 \sigma_u, 4.2 \sigma_u]$. Also following the authors, since the average debt levels for both countries do not exceed 150\%, the upper bound for debt is set as 1.5, therefore the debt range for the value function iteration is $[\underline{d}, \overline{d}]=[0,1.5]$.

I follow \citet{Schmitt-Uribe-16} and discretize the AR(1) process for output by constructing a transition probability matrix over the grids of the output.
A time series of 10 million observations was generated based on \refeq{eq:ar1-output}. Each observation is then assigned to the nearest grid point among the 200 discrete values of $\ln y^T$. The discretized series is analyzed to calculate the probabilities of transitioning from one discrete state to another in consecutive periods.
To obtain the transition probability matrix, a $200\times200$ matrix is initialized with zeros. For each pair of consecutive observations, the corresponding element in the matrix is incremented by 1. After considering all the observations, the matrix is normalized by dividing each row by the sum of its elements. This results in the estimated transition probability matrix, which effectively captures the covariance matrices of order 0 and 1
\citep*{Uribe-Schmitt-Grohe-textbook}.

Value function iteration is then conducted for a given set of calibrated parameters, which is composed of the set of fixed parameters $(\rho, \sigma_u, r^*, \theta, \alpha, a, \xi, \gamma)$ and the tuple $(\beta, \delta_1, \delta_2)$. The default dynamic and moments can then be conducted once the value function converges. Following \citet{Uribe-Schmitt-Grohe-textbook} and \citet{Na-18}, a simulation of the model based on the policy function is conducted sequentially\footnote{
    The main randomness comes from the AR(1) output. Once $y^T_{t+1}$ is determined, along with determination of the debt level for next period $d_{t+1}$, we have the state $(y^T_{t+1}, d_{t+1})$ for $t+1$, which is a grid-point. Values for the rest of the endogenous variables such as $c_{t+1}, I_{t+1}$ can then be obtained by extracting the corresponding value from the state grid-point on the policy function.
} for 1.1 million iterations, of which the first 0.1 million periods are discarded.

A critical procedure of the calibration is to select the tuple $(\beta, \delta_1, \delta_2)$ such that it matches the three targets described in section \ref{sec: calibration}, which are the default frequency, debt-to-GDP ratio, and output loss.
In particular, the tuple is chosen to minimize the Euclidean distance between the target value and the corresponding moments generated by the dynamic simulation\footnote{
    Default frequency is evaluated by taking the mean of $\left\{ I_{t} \right\}_{t \in T}$ and multiply it by $100\times 4$ quarters per century. Debt-to-GDP ratio is calculated by taking the mean of $\left\{ d_t / y^T_t \right\}_{t \in \text{Good T}}$, which is the average debt-to-tradable-output ratio condition on being in good standing. Output loss is evaluated by taking the mean of $\left\{ (y^T_t - \tilde{y}^T_t)/y^T_t \right\}_{t \in \text{Bad T}}$, which is the average output difference between the endowment and the actual received output, divided by the endowment, over periods of bad standings.
}. Since the value function iteration and dynamic simulation are time-consuming processes, I use the surrogate optimization algorithm in MATLAB as the searching algorithm, as it is suitable for minimizing time-consuming objective functions.

The estimated results are reported in Table~\ref{tab: calibration-compare}.
The upper part of the table presents the estimated values of $(\beta, \delta_1, \delta_2)$, and the lower part of the table shows the difference between the target and simulated moments under the optimal calibration (see the rows where the first column is ``HP'').
There exist errors in the minimal distance which is inevitable upon the computation. For Sri Lanka, the debt-to-tradable-GDP ratio is 1.73, lower than the target by 0.02, and the output loss is higher by 3\%. Frequency of default per century gives 1.26, which is about half the frequency I target. Similar results are obtained in the case of Pakistan. The debt-to-tradable-GDP ratio fits pretty well, and the output loss is lower by 1.3\%, but the frequency of default is also 1.26 times per century, also about have of the targeted default frequency.\footnote{
    With that being said, the results are actually pretty satisfying. The target of defaulting 2.6 times per century for both countries is set due to the ambiguity of determining default episodes, so I target a benchmark value of 2.6 times. However, Sri Lanka officially declared default for the first time in 2022, which yields $1/(2023-1948) \times 100 \approx 1.3$ times per century. As for Pakistan, \citet{Uribe-Schmitt-Grohe-textbook} and \citet{SPGlobal-default-report} only record the default event in 1999, which up to now yields $1/(2023-1947) \times 100 \approx 1.3$ times per century. These calculations are naive and inexact, but are roughly able to justify the estimated value of frequency of 1.26. A more robust target requires longer historical data throughout the history, which I am unable to tackle in this thesis.
}