Following \citet{Na-18}, the time unit is assumed to be one quarter, and the periodic utility function is assumed to be the constant relative risk aversion (CRRA) type
\begin{equation}
    \label{eq:CRRA-utility}
    U(c_t) = \frac{c_t^{1-\sigma} - 1}{1 - \sigma},
\end{equation}
where $\sigma$ is the inverse of elasticity of intertemporal substitution of the consumption.
The aggregator function for tradable and non-tradable consumption takes the constant elasticity of substitution (CES) form
\begin{equation}
    \label{eq:aggregator-function}
    c_t = A(c^T_t, c^N_t) =
        \left[ a \left( c^T_t \right)^{1- \frac{1}{\xi}} +
            (1 - a) \left( c^N_t \right)^{1- \frac{1}{\xi}}
        \right]^{\frac{1}{1 - \frac{1}{\xi}}}.
\end{equation}
The CES aggregator states that the share of tradable consumption is $a \in [0,1]$, and the elasticity of substitution between the tradable and non-tradable consumption is $\xi$.
Moreover, following the literature, to make the consumption of tradable goods $c^T_t$ and the external debt $d_t$ independent of the outputs in the nontradable sector in the equilibrium,
assume that the inter- and intratemporal elasticity of substitution is equivalent \citep*[See][Chapter 9.5]{Uribe-Schmitt-Grohe-textbook}.
That is,
\begin{equation}
    \label{eq:xi-sigma}
    \xi = \frac{1}{\sigma}.
\end{equation}
The production technology for the nontradable goods follows a simple form
\begin{equation}
    \label{eq:production-function}
    y^N_t = F(h_t) = h_t ^\alpha.
\end{equation}
The loss-function in \refeq{eq:ytt} is positive and increasing with $y^T_t$, and following \citet{Chatterjee-12}, I adopt the quadratic form with two parameters
\begin{equation}
    L(y^T_t) = \max \left\{
        0, \delta_1 y^T_t + \delta_2 \left( y^T_t \right)^2
     \right\}.
\end{equation}
This is also adopted in \citet{Na-18}. In this setting, if we set $\delta_1 < 0$ and $\delta_2 >0$, the output-loss increases as $y^T_t$ increases, indicating that the more a country is endowed, the more it loses during default.