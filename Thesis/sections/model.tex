International debt often lacks perfect enforcement, therefore the government holds the decision of whether to repay the debts or go default, based on the comparison of the future values \citep*{Eaton-Gersovitz-81}. Thus, default can be considered an optimal policy for a country.
If a country chooses to default, it faces the consequence of being excluded from the international credit market for a period of time and  would have to rely solely on its own financial resources. However, it would also benefit from not having to pay the interest on the debt.
Moreover, studies have pointed out that sovereign debt defaults are often accompanied by a devaluation of the currency; \citet*{Reinhart02} refers to this phenomenon as ``Twin Ds.''
Empirical analysis by \citet*{Na-18} further observes that the devaluation rate often decreases after the time of default, suggesting that the Twin Ds phenomenon is the joint result of an optimal policy.
They proposed a model that incorporates two key frictions: limited commitment to repay external debts and downward nominal wage rigidity.
The model predicts that default will occur only after a series of increasingly negative output shocks. Prior to default, domestic absorption experiences a severe contraction, which leads to a decline in demand for labor. However, due to downward nominal wage rigidity, real wages fail to adjust downward, resulting in involuntary unemployment. To prevent this situation, the optimal policy is to devalue the domestic currency, thereby reducing the real value of wages. As a result, both the model and the data show that default episodes are usually accompanied by significant currency devaluations \citep*{Na-18}.

For the sovereign debt model, I closely follow \citet*{Na-18}, as it allows me to match certain stylized facts about the sovereign debt defaults and examine the set of conditions where default is the optimal decision.
The calibrated model will then serve as a benchmark metric that allows us to examine whether China had set the heavily indebted poor counties into a default trap, following the method proposed by \citet*{Hinrichsen_2020-chapter4}.

\section{Households}
The model assumes that the economy is populated by a large number of representative households who maximize their expected lifetime utility
\begin{equation}
    \label{eq:utility}
    E_0 \sum_{t=0}^\infty \beta^t U(c_t),
\end{equation}
where $\beta \in(0,1)$ denotes the discount factor,
and $c_t$ represents the consumption good, which is composed of
tradable consumption $c_t^T$ and nontradable consumption $c_t^N$.
Assume that $c_t$ follows an aggregate technology
\begin{equation}
    \label{eq:A}
    c_t = A(c^T_t, c^N_t),
\end{equation}
where $A$ is an increasing, concave, and linearly homogeneous function that captures characteristics such as the ratio or elasticity of substitution between tradable and nontradable consumption.
The period utility function $U(c_t)$ follows the standard assumption, which is a strictly increasing and strictly concave function.

Assume that households only have access to the one-period and state non-contingent bond.
The households spend on consumption of tradable and untradable goods, along with their debt which is realized at this period. Their resources consist of labor incomes, dividend incomes, lump-sum transfers, as well as debt incomes. The households are also endowed with tradable goods, which follow a stochastic process.
The budget constraint of the representative household is then
\begin{equation}
    \label{eq:bc}
    P^T_t c^T_t + P^N_t c^N_t + P^T_t d_t =
    P^T_t \tilde{y}^T_t + W_t h_t + (1- \tau^d_t)P^T_t q^d_t d_{t+1} + F_t + \Phi_t,
\end{equation}
where $P^T_t (P^N_t)$ denotes the nominal price of tradable (nontradable) goods, $d_t$ the bond denominated in tradable goods which is due in period $t$, $q_t$ the price of debt to be repaid at $t+1$, $\tilde{y}^T_t$ the endowment of traded goods to the household, $W_t$ the nominal wage, $h_t$ the hours worked, $\tau^d_t$ the tax on debt, $F_t$ a lump-sum transfer from the government, and finally $\Phi_t$ the nominal profits from owning firms.
Households' working hour is bounded by an upper limit
\begin{equation}
    \label{eq:h-constraint}
    h_t \le \bar{h},
\end{equation}
and they take the working hours $h_t$ as given.

The households' problem is to choose $\{c_t, c_t^T, c_t^N, d_{t+1}\}$ such that their utility \eqref{eq:utility} is maximized subjected to the budget constraints \eqref{eq:A} -- \eqref{eq:h-constraint} and the no-Ponzi-game debt limit.
Further denote the relative price of nontradable in terms of tradable goods as $p_t \equiv \frac{P^N_t}{P^T_t}$, we have the following first order conditions
\begin{subequations}
    \begin{align}
        p_t &= \frac{A_2(c_t^T, c_t^N)}{A_1(c_t^t, c_t^N)}\\
        \lambda_t &= U'(c_t)A_1(c_t^T, c_t^N)\\
        (1-\tau_t^d)q_t^d \lambda_t &= \beta E_t \lambda_{t+1},
    \end{align}
\end{subequations}
where $\lambda_t$ is the Lagrange multiplier.

\section{Firms}
Perfectly competitive firms produce nontradable goods $y^N_t$ according to the production technology
\begin{equation}
    \label{eq:production}
    y^N_t = F(h_t),
\end{equation}
where $F$ is strictly increasing and strictly concave. Each firm maximizes its profit by choosing the amount of labor. Profit is given by
\begin{equation}
    \label{eq:profit}
    \Phi_t(h_t) = P^N_t F(h_t) - W_t h_t,
\end{equation}
and the optimal labor demand is then
\begin{equation*}
    P^N_t F'(h_t) = W_t.
\end{equation*}
Dividing both side by the price of tradable goods, and define $w_t \equiv \frac{W_t}{P^T_t}$ as the real wage in terms of tradable goods, the first order condition can be written as
\begin{equation}
    \label{eq:firm-FOC}
    p_t F'(h_t) = w_t.
\end{equation}

\section{Downward Nominal Wage Rigidity}
The key assumption in \citet*{Schmitt-Uribe-16} and \citet*{Na-18} is the downward nominal wage rigidity.
As the wage is unable to be adjusted to a lower level, involuntary unemployment is inevitable, hence the government has the incentive to allow devaluation. The model imposes a lower bound to the growth rate of nominal wage
\begin{equation}
    W_t \ge \gamma W_{t-1}, \qquad \gamma > 0.
\end{equation}
This implies that the growth rate $\frac{W_{t} - W_{t-1}}{W_{t-1}} \ge \gamma - 1$. When this inequality is unbinding ($W_t > \gamma W_{t-1}$), the economy is fully employed ($h_t = \bar{h}$). However, if the condition binds, the economy might have unemployment ($h_t < \bar{h}$). This relationship can be written as the following equation
\begin{equation}
    \label{eq:wage-rigid}
    (\bar{h} - h_t)(W_t - \gamma W_{t-1}) = 0.
\end{equation}


\section{Government}
I assume here, under the lack of enforcement in the international credit market that the government has the option to benevolently free up its domestic balance sheet by choosing to default or not.
Denote $I_t$ as the indicator of whether the government chooses to honor its debt in period $t$. If the government repays in this period ($I_{t} = 1$), then the country will be able to borrow in the following period, and hence $d_{t+1} > 0$. However, if the government chooses to default ($I_t = 0$), then the country will enter the status of financial autarky and is unable to have any sovereign debt in the next period, hence $d_{t+1} = 0$. The above scenario can be written as a slackness condition:
\begin{equation}
    \label{eq:gov-next-debt}
    (1 - I_t)d_{t+1} = 0 .
\end{equation}

To model the duration of financial exclusion, assume that once the country is in bad standing in the international credit market, it can regain a fiscally sound reputation and access to financial markets with probability $\theta \in [0,1)$, and remain in bad standing with probability $1-\theta$. This implies that the country has an average exclusion duration of $\frac{1}{\theta}$ periods.\footnote{
    The expected exclusion period $= \sum_{t=1}^{\infty} t \theta (1-\theta)^{t-1} = \theta  \sum_{t=1}^{\infty} t (1-\theta)^{t-1} = \frac{1}{\theta}$.
}

Assume that the government distributes the proceeds from the debt tax to households as a lump-sum payment. If the government honors the debt, then it repays $d_t$, but if the government decides to default, then it will not make any payments to foreign lenders, and instead will return any payments made by households directly to them. The budget constraint for the government can then be expressed as:
\begin{equation}
    \label{eq:gov-budget}
    f_t = \tau_t^d q_t^d d_{t+1} + (1-I_t)d_t,
\end{equation}
where $f_t \equiv \frac{F_t}{P^T_t}$ is the lump-sum transfer in terms of tradable goods. The right-hand side of the equation states that the transfer to households will include $d_t$ when $I_t = 0$, which is when the country decides to default. Nevertheless, the transfer of debt tax will be zero after default since $d_{t+1} = 0$ when $I_t = 1$, according to Equation \eqref{eq:gov-next-debt}.
