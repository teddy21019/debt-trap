So far, most discussions on China's debt trap have been limited to narrative statistics.
To objectively assess this issue, we adopt the up-to-date sovereign debt default model in the literature as a tool for our empirical analysis. \citet{Na-18} proposes a model to study the Argentine economy; \citet*{Hinrichsen_2020-chapter4} uses the model to study the enforcement of sovereign debt under war reparations.
The model in my thesis strictly follows \citet{Na-18} and \citet*{Hinrichsen_2020-chapter4}.

International debt often lacks enforcement, and governments hold the decision of whether to repay the debt or default, based on the comparison of future values \citep*{Eaton-Gersovitz-81}. Therefore, default can be considered an optimal policy for a country that faces unsustainable debt levels. By defaulting, the country avoids the burden of paying interest on the debt, but it also faces the consequence of being excluded from the international credit market for a period of time. As a result, the country would have to rely solely on its own financial resources until it regains access to international credit markets.
Moreover, studies have pointed out that sovereign debt defaults are often accompanied by a devaluation of the currency; \citet*{Reinhart02} refers to this phenomenon as ``Twin Ds.''
Empirical analysis by \citet{Na-18} further observes that the devaluation rate often decreases after the time of default, suggesting that the Twin Ds phenomenon is the joint result of an optimal policy.
They proposed a model that incorporates two key frictions: limited commitment to repay external debts and downward nominal wage rigidity.
It is a decentralized version of the Eaton-Gersovitz sovereign debt model.
The model predicts that default will occur only after a series of increasingly negative output shocks. Prior to default, domestic absorption experiences a severe contraction, which leads to a decline in demand for labor. However, due to downward nominal wage rigidity, real wages fail to adjust downward, resulting in involuntary unemployment. To prevent this situation, the optimal policy is to devalue the domestic currency, thereby reducing the real value of wages. As a result, both the model and the data show that default episodes are usually accompanied by significant currency devaluations \citep*{Na-18}.

Therefore, for the sovereign debt model, I closely follow \citet{Na-18} to examine the set of conditions under which default is the optimal decision.
The calibrated model will then serve as a benchmark metric that allows us to investigate whether China has potentially trapped heavily indebted poor counties into default, using the approach proposed by \citet*{Hinrichsen_2020-chapter4}.

\section{Households}
The model assumes that the economy is populated by a large number of representative households who maximize their expected lifetime utility
\begin{equation}
    \label{eq:utility}
    E_0 \sum_{t=0}^\infty \beta^t U(c_t),
\end{equation}
where $\beta \in(0,1)$ denotes the discount factor,
and $c_t$ represents the consumption good, which is composed of
tradable consumption $c_t^T$ and nontradable consumption $c_t^N$.
Assume that $c_t$ follows an aggregate technology
\begin{equation}
    \label{eq:A}
    c_t = A(c^T_t, c^N_t),
\end{equation}
where $A$ is an increasing, concave, and linearly homogeneous function that captures characteristics such as the ratio or elasticity of substitution between tradable and nontradable consumption.
The period utility function $U(c_t)$ follows the standard assumption, which is a strictly increasing and strictly concave function.

Assume that households only have access to the one-period and state non-contingent bond.
The households spend on consumption of tradable and untradable goods, along with their debt which is realized at this period. Their resources consist of labor incomes, dividend incomes, lump-sum transfers, as well as debt incomes. The households are also endowed with tradable goods, which follow a stochastic process.
The budget constraint of the representative household is then
\begin{equation}
    \label{eq:bc}
    P^T_t c^T_t + P^N_t c^N_t + P^T_t d_t =
    P^T_t \tilde{y}^T_t + W_t h_t + (1- \tau^d_t)P^T_t q^d_t d_{t+1} + F_t + \Phi_t,
\end{equation}
where $P^T_t (P^N_t)$ denotes the nominal price of tradable (nontradable) goods, $d_t$ the bond denominated in tradable goods which is due in period $t$, $q_t$ the price of debt to be repaid at $t+1$, $\tilde{y}^T_t$ the endowment of traded goods to the household, $W_t$ the nominal wage, $h_t$ the hours worked, $\tau^d_t$ the tax on debt, $F_t$ a lump-sum transfer from the government, and finally $\Phi_t$ the nominal profits from owning firms.
Households' working hour is bounded by an upper limit
\begin{equation}
    \label{eq:h-constraint}
    h_t \le \bar{h},
\end{equation}
and they take the working hours $h_t$ as given.

The households' problem is to choose $\{c_t, c_t^T, c_t^N, d_{t+1}\}$ such that their utility \eqref{eq:utility} is maximized subjected to the budget constraints \eqref{eq:A} -- \eqref{eq:h-constraint} and the no-Ponzi-game debt limit.
Further denote the relative price of nontradable in terms of tradable goods as $p_t \equiv \frac{P^N_t}{P^T_t}$, we have the following first order conditions
\begin{subequations}
    \begin{align}
        p_t &= \frac{A_2(c_t^T, c_t^N)}{A_1(c_t^t, c_t^N)}\\
        \lambda_t &= U'(c_t)A_1(c_t^T, c_t^N)\\
        (1-\tau_t^d)q_t^d \lambda_t &= \beta E_t \lambda_{t+1},
    \end{align}
\end{subequations}
where $\lambda_t$ is the Lagrange multiplier.

\section{Firms}
Perfectly competitive firms produce nontradable goods $y^N_t$ according to the production technology
\begin{equation}
    \label{eq:production}
    y^N_t = F(h_t),
\end{equation}
where $F$ is strictly increasing and strictly concave. Each firm maximizes its profit by choosing the amount of labor. Profit is given by
\begin{equation}
    \label{eq:profit}
    \Phi_t(h_t) = P^N_t F(h_t) - W_t h_t,
\end{equation}
and the optimal labor demand is then
\begin{equation*}
    P^N_t F'(h_t) = W_t.
\end{equation*}
Dividing both side by the price of tradable goods, and define $w_t \equiv \frac{W_t}{P^T_t}$ as the real wage in terms of tradable goods, the first order condition can be written as
\begin{equation}
    \label{eq:firm-FOC}
    p_t F'(h_t) = w_t.
\end{equation}

\section{Downward Nominal Wage Rigidity}
The key assumption in \citet*{Schmitt-Uribe-16} and \citet*{Na-18} is the downward nominal wage rigidity.
As the wage is unable to be adjusted to a lower level, involuntary unemployment is inevitable, hence the government has the incentive to allow devaluation. The model imposes a lower bound to the growth rate of nominal wage
\begin{equation}
    W_t \ge \gamma W_{t-1}, \qquad \gamma > 0.
\end{equation}
This implies that the growth rate $\frac{W_{t} - W_{t-1}}{W_{t-1}} \ge \gamma - 1$. When this inequality is unbinding ($W_t > \gamma W_{t-1}$), the economy is fully employed ($h_t = \bar{h}$). However, if the condition binds, the economy might have unemployment ($h_t < \bar{h}$). This relationship can be written as the following equation
\begin{equation}
    \label{eq:wage-rigid}
    (\bar{h} - h_t)(W_t - \gamma W_{t-1}) = 0.
\end{equation}


\section{Government}
I assume here, under the lack of enforcement in the international credit market that the government has the option to benevolently free up its domestic balance sheet by choosing to default or not.
Denote $I_t$ as the indicator of whether the government chooses to honor its debt in period $t$. If the government repays in this period ($I_{t} = 1$), then the country will be able to borrow in the following period, and hence $d_{t+1} > 0$. However, if the government chooses to default ($I_t = 0$), then the country will enter the status of financial autarky and is unable to have any sovereign debt in the next period, hence $d_{t+1} = 0$. The above scenario can be written as a slackness condition:
\begin{equation}
    \label{eq:gov-next-debt}
    (1 - I_t)d_{t+1} = 0 .
\end{equation}

To model the duration of financial exclusion, assume that once the country is in bad standing in the international credit market, it can regain a fiscally sound reputation and access to financial markets with probability $\theta \in [0,1)$, and remain in bad standing with probability $1-\theta$. This implies that the country has an average exclusion duration of $\frac{1}{\theta}$ periods.\footnote{
    The expected exclusion period $= \sum_{t=1}^{\infty} t \theta (1-\theta)^{t-1} = \theta  \sum_{t=1}^{\infty} t (1-\theta)^{t-1} = \frac{1}{\theta}$.
}

Assume that the government distributes the proceeds from the debt tax to households as a lump-sum payment. If the government honors the debt, then it repays $d_t$, but if the government decides to default, then it will not make any payments to foreign lenders, and instead will return any payments made by households directly to them. The budget constraint for the government can then be expressed as:
\begin{equation}
    \label{eq:gov-budget}
    f_t = \tau_t^d q_t^d d_{t+1} + (1-I_t)d_t,
\end{equation}
where $f_t \equiv \frac{F_t}{P^T_t}$ is the lump-sum transfer in terms of tradable goods. The right-hand side of the equation states that the transfer to households will include $d_t$ when $I_t = 0$, which is when the country decides to default. Nevertheless, the transfer of debt tax will be zero after default since $d_{t+1} = 0$ when $I_t = 1$, according to Equation \eqref{eq:gov-next-debt}.

\section{Foreign Lenders}
The behavior of foreign lenders is not explicitly modeled in this framework, but as all rational agents, the expected marginal utility of lending to the domestic country must be equivalent to the opportunity cost of funds.
Let $r^*$ represent the opportunity cost for the foreign lenders; this could be the world interest rate. Since $q_t$ is the price of debt that repays one unit of $d_{t+1}$ tomorrow, the return on the debt is $\frac{1}{q_t}$. The lenders take the risk of default into consideration, and hence the expected return will actually be lower. Assume that foreign lenders are risk neutral, this gives
\begin{equation}
    \label{eq:lender}
    \frac{\Pr(I_{t+1}=1 \mid I_{t}=1)}{q_t} = 1 + r^* .
\end{equation}

\section{Competitive Equilibrium}
Under equilibrium, households' consumption equals the production of firms:
\begin{equation}
    \label{eq:nontrade-clear}
    c^N_{t} = y^N_t.
\end{equation}
The tradable goods are purely endowed exogenously under an AR(1) process:
\begin{equation}
    \label{eq:ar1-output}
    \ln(y_t^T) = \rho \ln(y^T_{t-1}) + \mu_t,
\end{equation}
where $\mu_t \overset{\mathrm{iid}}{\sim} \mathcal{N}(0,\sigma_\mu^2)$ is an i.i.d. shock, and $ |\rho| \in [0,1)$ is the autocorrelation parameter.
When the country decides to default, it is in bad standing, and hence it faces an output loss defined by $L(y^T_t)$. The loss function is non-negative and increasing in the tradable goods. The endowment of tradable goods to the household is then:
\begin{equation}
    \label{eq:ytt}
    \tilde{y}^T_t =
        \begin{cases}
        y^T_t  - L(y^T_t) & \text{if } I_t = 0 \\
        y^T_t & \text{otherwise.}
        \end{cases}
\end{equation}
When the country defaults ($I_t = 0$), the endowment decreases.

The price of debt offered by foreign lenders $q_t$ should equal the price of the domestic debt $q^d_t$, but only during good standing:
\begin{equation}
    \label{eq:qq}
    I_t(q^d_t - q_t) = 0.
\end{equation}

The market-clearing condition can be established by combining various equations, including household budget constraints \eqref{eq:bc} and \eqref{eq:h-constraint}, the firm's production function \eqref{eq:production} and profit equation \eqref{eq:profit}, the government's constraint on debt \eqref{eq:gov-next-debt} and lump-sum return \eqref{eq:gov-budget}, and the conditions from \eqref{eq:nontrade-clear}, \eqref{eq:ytt}, and \eqref{eq:qq}.
Eventually, the clearing condition for tradable goods is:
\begin{equation}
    \label{eq:market-clearing}
    c^T_t = y^T_t - (1 - I_t)L(y^T_t) + I_t(q_t d_{t+1} - d_t)
\end{equation}

Assume that the law of one price applies to tradable goods. The foreign currency price of tradable goods is denoted as $P^{T*}_t$, while the nominal exchange rate is represented by $\mathcal{E}_t$. The law of one price states that the price of tradable goods in the domestic currency is equal to the foreign currency price multiplied by the nominal exchange rate.
\begin{equation*}
    P^T_t = P^{T*}_t \mathcal{E}_t
\end{equation*}
This implies that the price of a tradable good should be the same in both domestic and foreign currency terms in an efficient market.

Without loss of generosity, the foreign-currency price of the tradable goods is normalized to 1 ($P^{T*}_t = 1$)
Hence, the nominal price for tradable goods can be expressed as the nominal exchange rate:
\begin{equation}
    \label{eq:price-exrate}
    P^T_t = \mathcal{E}_t.
\end{equation}
For convenience, I also define the devaluation rate of domestic currency as:
\begin{equation}
    \label{eq:devaluation-rate}
    \epsilon_t \equiv \frac{\mathcal{E}_t}{\mathcal{E}_{t-1}} = \frac{P^T_t}{P^T_{t-1}}.
\end{equation}
The conditions are now sufficient to define a competitive equilibrium.
\begin{definition}[Competitive Equilibrium in \citet{Na-18}]
    A competitive equilibrium is a set of stochastic processes $\left\{ c^T_t, h_t, w_t, d_{t+1}, \lambda_t, q_t, q^d_t \right\}$ satisfying:
    \begin{align}
    c^T_t &= y^T_t - (1 - I_t)L(y^T_t) + I_t(q_t d_{t+1} - d_t), \\
    (1 - I_t)d_{t+1} &= 0, \\
    \lambda_t &= U'(A(c^T_t, F(h_t)))A_1(c_t^T, c_t^N),\\
    (1-\tau_t^d)q_t^d \lambda_t &= \beta E_t \lambda_{t+1}, \\
    I_t(q^d_t - q_t) &= 0, \\
    \frac{A_2(c_t^T, F(h_t))}{A_1(c_t^t, F(h_t))} &= \frac{w_t}{F'(h_t)} , \\
   w_t &\ge \gamma\frac{w_{t-1}}{\epsilon_t},\\
   h_t &\le \bar{h},\\
   \left( h_t - \bar{h} \right) \left( w_t - \gamma\frac{w_{t-1}}{\epsilon_t}\right) &= 0, \\
    I_t \left[ q_t - \frac{E_t I_{t+1}}{1+r^*} \right] &= 0,
\end{align}
    given processes $\left\{ y^T_t, \epsilon_t, \tau^d_t, I_t \right\}$ and initial conditions $w_{-1}$ and $d_0$.
\end{definition}
As proven by \citet{Na-18}, if the government is able to set the devaluation rate and the tax on debt freely, then the stochastic process of the variables $\left\{ c^T_t, h_t, d_{t+1}, q_t \right\}$ can be determined by the process of $\left\{ y^T_t, I_t\right\}$ and the initial debt level $d_0$.

As discussed previously, the decision of $I_t$ is an optimal policy for the government due to a lack of commitment to repay debt in the international credit market. Furthermore, the default decision of the government in the next period $t+1$ is also affected by the current decision. To see this argument, first notice that the default decision in $t+1$ is determined by the state variables $\left\{ y^T_{t+1}, d_{t+1} \right\}$. However, $d_{t+1}$ is determined in period $t$, which means that the government in period $t$ understands that it is able to affect the default decision in $t+1$ via the choice of $d_{t+1}$. As $y^T_{t+1}$ follows a first-order Markov process, the expected value of $y^T_{t+1}$ is a function of $y^T_t$, and hence the expected value for the default decision on period $t$ is actually a function of $y^T$ and $d_{t+1}$.

Recall that the price for debt $q_t$ relates to the probability of default in the next period. According to Equation \eqref{eq:lender}, it can be expressed in the contemporary variables:
\begin{equation}
    q_t = q(y^T_t, d_{t+1}).
\end{equation}
On the one hand, this provides economic intuition that the government internalizes the fact that its choice of debt in the next period can affect the price of the debt. On the other hand, this clarifies the dependencies of variables in the value function.


\section{Default Decision}
Following the standard Eaton-Gersovitz framework, this model considers the following three value functions:
value of continuing to repay the debt $v^c$, value of being in good standing $v^g$, and value of being in bad standing $v^b$.

Under the period of being in good financial standing, the value for the government to continue repaying the debt is the maximum value of the utility gained by the households this period, plus the discounted value of being in a good financial standing, subject to the households' budget constraints. Formally,
\begin{equation}
    \begin{aligned}
        v^c(y^T_t, d_t) = \max_{\left\{ c^T_t, h_t, d_{t+1} \right\}} \quad
        &\left\{
            U\left(
                A\left(c^T_t, F(h_t)\right)
             \right)
             + \beta E_t
             v^g \left(
                y^T_{t+1}, d+{t+1}
              \right)
         \right\}\\
          \text{s.t} \quad& c^T_t + d_t = y^T_t + q(y^T_t, d_{t+1}) d_{t+1} \\
                    & h_t \le \bar{h}.
    \end{aligned}
\end{equation}
Where the first constraint is obtained by setting $I_t = 1$ in \refeq{eq:market-clearing}, and the second is the constraint on working hour.

If the country is in bad standing, the consumption on tradable goods experiences a loss. The government has probability $\theta$ of regaining access to international financial markets, and probability $1 - \theta$ of continuing in bad standing. During the period in bad standing, the country obtains no international borrowing, hence, the state variable for debt is excluded. Formally,
\begin{equation}
    \begin{aligned}
        v^b(y^T_t) = \max_{\left\{ h_t \right\}} \quad
        &\left\{
            U\left(
                A\left( y^T_t - L(y^T_t), F(h_t)\right)
             \right)
             + \beta E_t \left[
                \theta v^g \left(
                    y^T_{t+1}, 0
                \right)
                + (1-\theta) v^b \left(
                    y^{T}_{t+1}
                 \right)
            \right]
         \right\}\\
          \text{s.t} \quad& h_t \le \bar{h}.
    \end{aligned}
\end{equation}
The tradable consumption $c^T_t = y^T_t - L(y^T_t)$ again follows \refeq{eq:market-clearing} by setting $I_t = 0$, and is substituted explicitly into the value function.

If the country is in good standing, the government has the freedom to choose which is best for the country: to continue or to default. The decision is made by comparing the value functions of the two scenarios, given the current output shock for tradable goods and the current level of debt
\begin{equation}
    v^g(y^T_t, d_t) = \max\left\{
        v^c(y^T_t, d_t) ,
        v^b(y^T_t)
     \right\}.
\end{equation}

Define the default set $D(d_t)$ as the set of tradable-output levels $y^T_t$ examined by the government in period $t$, in which the government's optimal respond is to default. Formally,
\begin{equation}
    D(d_t) = \left\{ 
        y^T_t : v^b(y^T_t) > v^c(y^T_t, d_t)
     \right\}.
\end{equation}
In other words, given a current debt level $d_t$, if the government observes that $y^T_t$ is inside $D(d_t)$, it chooses to default.

Under rational expectations, the foreign lenders recognize the default set, hence the price for debt is determined by Equation \eqref{eq:lender}, given by
\begin{equation}
    q(y^T_t, d_{t+1}) =
    \frac{1 - \Pr\left\{ y^T_{t+1} \in D(d_{t+1}) \mid y^T_t \right\}}{1 + r^*}.
\end{equation}
Note that the price of debt enters the value function of continuing, $v^c(y^T_t, d_{t})$.

It is obvious that the optimal labor supply is $h_t = \bar{h}$ since all functions, $F, A, U$, are monotonic, which implies that under the freedom to choose the devaluation rate and the tax on debt, the government can ensure full employment. Denote $w^f(c^T_t)$ the equilibrium wage function under full employment given the consumption of tradable goods. Combining Equation \eqref{eq:firm-FOC} and the Euler equation in \eqref{eq:FOC-HH-1} and impose the optimal policy $h_t = \bar{h}$, we have
\begin{equation}
    w_t = w^f(c^T_t) \equiv \frac{A_2(c^T_t, F(\bar{h}))}{A_1(c^T_t, F(\bar{h}))} F'(\bar{h}).
\end{equation}
Knowing that the wage has downward nominal rigidity, the government sets the devaluation rate accordingly. The downward rigidity \eqref{eq:wage-rigid} states that
\begin{equation*}
    \gamma \le \frac{W_t}{W_{t-1}} = \frac{w_t}{w_{t-1}} \frac{P^T_t}{P^T_{t-1}} = \epsilon \frac{w_t}{w_{t-1}},
\end{equation*}
where the second equal sign comes from Equation \eqref{eq:devaluation-rate}. Substitute the wage under full employment, we get
\begin{equation}
    \epsilon_t \ge \gamma \frac{w_{t-1}}{w^f(c^T_t)}.
\end{equation}
This is the family of optimal devaluation policies. Following \citet{Na-18} and \citet*{Hinrichsen_2020-chapter4}, we assume that the government chooses the minimal devaluation target that stabilizes nominal wages, that is, $
    \epsilon_t = \gamma \frac{w_{t-1}}{w^f(c^T_t)}.
$

