\begin{table}[t]
\centering
\begin{tabular}{@{}lll@{}}
Description                      & Source                             & Code/Method               \\ \midrule
Real GDP on Agriculture          & World Bank                         & NV.AGR.TOTL.KD            \\
Real GDP on Industry             & World Bank                         & NV.IND.TOTL.KD            \\
Real Tradable GDP                &                                    & Sum of the above two data \\
Real GDP                         & World Bank                         & NY.GDP.MKTP.KD            \\
Nominal GDP on Agriculture       & World Bank                         & NV.AGR.TOTL.CD            \\
Nominal GDP on Industry          & World Bank                         & NV.IND.TOTL.CD            \\
Nominal Tradable GDP             &                                    & Sum of the above two data \\
Nominal GDP                      & World Bank                         & NY.GDP.MKTP.CD            \\
Population                       & World Bank                         & SP.POP.TOTL               \\
Nominal Trade Balance            & World Bank                         & BN.GSR.MRCH.CD            \\
External Debt to All Creditors & International Debt Statistics      &                           \\
External Debt to China (part)  & International Debt Statistics      &                           \\
External Debt to China         & Horn, Reinhart and Trebecsh (2021) & \\\bottomrule
\end{tabular}
\caption{Data Source}
\label{tab: data-source}
\floatfoot{\emph{Note:} Real GDP refers to using data in constant 2015 U.S. dollar. The sector name ``agriculture, forestry, and fishing'' is simplified as ``Agriculture'' in the table. ``Industry'' consists of mining, manufacturing, construction, electricity, water, and gas.
External debt to China has two sources: International Debt Statistics and \citet*{Horn-Reinhart-Trebesch-21}. In the empirical result throughout the thesis, the main data for the debt to China is from \citet*{Horn-Reinhart-Trebesch-21} as it includes loans from state-owned commercial banks as well. Data for the debt to China reported in International Debt Statistics does not contain this category, hence is marked ``part'' in the table.

}
\end{table}