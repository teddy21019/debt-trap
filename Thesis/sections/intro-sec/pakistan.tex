Pakistan's economic trajectory has also been characterized by a series of fluctuations throughout its history. In the period from the 1980s to 1996, the country achieved an average GDP growth rate of around 5\%. However, Pakistan faced significant challenges when it conducted nuclear tests in May 1998, leading to an embargo imposed by the United States. This, along with other factors, resulted in a default in 1999. Inflation has remained a persistent concern, with an average rate of 8.2\%. During the global financial crisis of 2008, inflation soared to 20\% while GDP growth remained sluggish at 2\%.

Natural calamities, including the devastating floods of 2010 and 2011, further dampened the economy, resulting in a meager GDP growth rate of 1.5\%. Nevertheless, Pakistan experienced gradual economic recovery after 2008, with the GDP showing slow but steady improvement and inflation gradually subsiding. By 2015, inflation had been successfully controlled at 2.5\%. However, the COVID-19 pandemic dealt a severe blow to the economy. As of 2022, Pakistan's economic landscape is characterized by the following sectoral contributions to the GDP: services account for 51.5\%, industry for 19.8\%, and agriculture for 22.3\%.

Similar to Sri Lanka, Pakistan has also experienced a significant surge in its debt to China, while simultaneously maintaining a relatively high debt level with other official creditors. As depicted in Figure \ref{fig: pakistan-debt-ts}, the composition of creditors to Pakistan has undergone notable changes.  Beginning in 2013, the loans from China escalated from \$4 billion to \$13 billion in 2014, and further rose to \$27.6 billion in 2017, more than doubling the amount in 2014.

% intro pak economy: GDP+ inflation, flood, ..etc

Pakistan is the centerpiece of the China Pakistan Economic Corridor (CPEC), a 3000 km corridor that connects China with the Arabian Sea.
CPEC is an important network as it reduces the passage for China's energy import from the Middle Eastern countries \citep*{CPEC-wiki}.
It serves as a strategic trade shortcut, connecting the Middle East and Europe while bypassing the Malacca Straits. The corridor originates from Kashgar in China's Xinjian province, passes through the Gate of Khunjerab, traverses Pakistan's capital city Islamabad, and extends all the way to Karachi, where Pakistan meets the Arabic sea. At the terminus of the corridor lies Gwadar, which provides access to the Oman Gulf.
In order to accommodate the needs for CPEC, China has launched enormous amounts of infrastructure in Pakistan after 2015. These items include a deep water port, road and rail lines, and most importantly, energy sector projects.

Massive amount of commercial loans are provided to Pakistan in 2016, with most of them focusing on power projects.
These projects include:
\$2.7 billion on the Gwadar-Nawabshah LNG terminal and pipeline project;
\$1.26 billion on the Karot Hydropower Project;
\$1.28 billion on the Matiari to Lahore Transmission Line;
\$1.55 billion on the Pakistan Port Qasim Power Project; and \$0.75 billion on the Qasim Datang Power Station \citep*{Horn-Reinhart-Trebesch-21}.

In 2017, another \$3 billion on power projects are lent, including \$1.35 billion for Suki Kinari Hydropower Project and \$1.5 billion on the Hubco Coal Power Plant Project.
Up to 2018, the estimation of the total value of projects under the CPEC is \$62 billion, out of which around \$33 billion is allocated for energy projects~\citep*{Hurley19-8-debt-trap}. China is expected to finance about 80\% of this amount. The private investments for energy projects in Pakistan will be financed by the Exim Bank of China at an interest rate of 5-6\%.

Gwadar Port, located in the western corner of Pakistan, is another prominent component of the CPEC.
Recognizing its pivotal location, China has planned to expand the port and create additional berths. In November 2015, the China Overseas Port Holding Company secured a 43-year lease for Gwadar, enabling port expansion and the establishment of a "Special Economic Zone," similar to Sri Lanka's situation \citep{Ranade-17-CPEC}.
The level of debt resulting from this project, however, remains uncertain. A governor of the Central Bank of Pakistan once admitted that he lacked precise information regarding the composition of the \$46 billion investment, stating, ``I don't know out of the \$46 billion how much is debt, how much is equity, and how much is in kind \citep{small2020returning}.''

Consequently, the total debt in Pakistan is at least \$130 billion, which yields about 37.5\% debt-to-GDP ratio.\footnote{
    As with the reason of Sri Lanka, debt data retrieved from International Debt Statistics is underestimated since it does not include state-owned commercial banks.
}
The sudden increase of debt to China draws the attention of researchers and journalists. For example, a report from the Financial Time titled ``Pakistan is on the brink'' states that Pakistan is following Sri Lanka into default.
Given the recent frequent analogy drawn between Pakistan and Sri Lanka, it is essential to analyze Pakistan from the perspective of the sovereign default model.
