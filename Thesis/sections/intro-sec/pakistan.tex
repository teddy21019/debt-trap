Similar to Sri Lanka, we observe the abrupt increase on the debt to China, as it already has a relatively high debt to other official creditors. \autoref{fig: pakistan-debt-ts} shows the change in composition of creditors to Pakistan. 
Pakistan is the centerpiece of the China Pakistan Economic Corridor (CPEC), a 3000 km corridor that connects China with the Arabian sea.
CPEC serves as an important network as it reduces the passage for China's energy import from the Middle Eastern countries \citep*{CPEC-wiki}.

China has launched enormous amounts of infrastructure in Pakistan after 2015. These items include a deep water port, road and rail lines, and most importantly, energy sector projects. Up to 2018, the estimation of the total value of projects under the CPEC is \$62 billion, out of which around \$33 billion is allocated for energy projects~\citep*{Hurley19-8-debt-trap}. China is expected to finance about 80 percent of this amount. The private investments for energy projects in Pakistan will be financed by the Exim Bank of China at an interest rate of 5-6\%. 
Private Independent Power Producers (IPP) will be responsible for constructing the energy projects under CPEC, instead of the governments of China or Pakistan. In turn, the government of Pakistan will be legally bound to buy electricity from these companies at rates that were agreed upon before.
However, despite this significant investment, some projects have already been cancelled, such as three major road projects that were cancelled at the end of 2017 \citep*{Hurley19-8-debt-trap}.

In the case of Pakistan, the sudden increase of debt to China draws the attention of researchers and journalists. For example, a report from the Financial Time titled ``Pakistan is on the brink'' states that Pakistan is following Sri Lanka into default. Given the recent frequent analogy drawn between Pakistan and Sri Lanka, it is essential to analyze Pakistan from the perspective of the sovereign default model.
