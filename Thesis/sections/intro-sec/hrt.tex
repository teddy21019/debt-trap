International debt to China lacks transparency since their official lending is predominantly undertaken by state-owned entities, and unlike other major economies, the Chinese government does not report or publish any data on its official international lending or outstanding overseas debt claims \citep*{Horn-Reinhart-Trebesch-21}.

% China's official external lending is predominantly undertaken by state-owned entities and the government itself\footnotemark{}. However, unlike other major economies, the Chinese government does not report or publish any data on its official international lending or outstanding overseas debt claims. This lack of transparency creates challenges for rating agencies, as official lending to sovereigns is not a regular part of their activities. Moreover, China is not a member of the Paris Club, which tracks sovereign borrowing from official bilateral creditors, and does not divulge data on its official flows with the OECD's Creditor Reporting System \citep*{Horn-Reinhart-Trebesch-21}.
% \footnotetext{These include China's state-owned policy banks, such as China Development Bank (國家開發銀行, CDB) and China Export-Import Bank (中國進出口銀行, Exim), as well as China's state-owned commercial banks such as Industrial and Commercial Bank of China (中國工商銀行, ICBC) or Bank of China (中國銀行, BoC)}

\citet*{Horn-Reinhart-Trebesch-21} combines a variety of sources to construct a consensus database of Chinese official loans.
Their database spans from 1949, the establishment of the People's Republic of China, to 2017. It contains a granular dataset of 2151 loans and 2824 grants with information such as the creditor agent, borrower type, commitment, maturity, etc. It also provides an aggregate panel data of the external debt to China for each country.
To have a more precise data of the China debt issue, the database constructed by \citet*{Horn-Reinhart-Trebesch-21} is utilized as the primary source of Chinese debt data. This database is combined with data from other creditors obtained from the International Debt Statistics (IDS) from World Bank.

The top 30 countries with the largest debts to China's official creditors are displayed in \autoref{fig:total-debt-30}. Notably, Russia owes China over \$70 billion, while Pakistan's debt amounts to \$27 billion, both topping the list. Brazil and Venezuela are among the top 10 countries with the highest debt to China in Latin America. Contrary to what many people believe, African countries have not borrowed much from China. However, if we consider the ratio of Chinese debt to GDP in \autoref{fig:perc-debt-30}, some African countries appear to be highly indebted to China. Djibouti, for instance, has an alarming ratio of 68.5\% of its GDP consisting of Chinese debt, while Tonga, Niger, and Zambia have ratios exceeding 10\%.

A main finding in \citet*{Horn-Reinhart-Trebesch-21} is that China had become the world's largest creditor to developing countries after 2013, surpassing the amount of World Bank. \autoref{fig:debt-ts} shows the change of the total amount of debt from different main creditors, including China, World Bank\footnote{Including the International Development Association (IDA) and the International Bank for Reconstruction and Development (IBRD) }, IMF, and the aggregation of all countries in the Paris Club. The debt amount started to rise rapidly after 2000, when the China government launched the ``Go Out Policy'' in 1999. In 2017, the debt to China in global had reached \$355 billion, while the debt to World Bank was \$300 billion.

The main focus on my thesis is to evaluate the debt sustainability for a country after it receives the considerable amount of loans from China. Among all countries, Sri Lanka and Pakistan are often under the discussion regarding the debt-trap issue. From the geostrategic aspect, Sri Lanka could serve as the military base for China \citep{Chellaney_2017}, and Pakistan allows China to better connect with the Arabian Sea \citep{Hurley19-8-debt-trap}. A preliminary view of the change in debt to China for the two countries in Figure \ref{fig: LAK-PAK-debt-ts} also provides some idea of how China is accumulating an increasing amount of debt to the two country. I briefly introduce the debt-trap diplomacy narrative regarding the two countries in the following sections.