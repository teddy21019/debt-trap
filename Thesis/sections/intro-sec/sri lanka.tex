Sri Lanka's economic situation has been marked by various challenges and fluctuations throughout its history. From 1983 to 2009, the country endured a civil war that significantly impacted its economy. Despite this, Sri Lanka managed to achieve an average GDP growth rate of approximately 4\%. Following the end of the civil war, its GDP experienced notably even stronger growth, reaching 7 to 8\%. However, the rate slowed down in subsequent years. Inflation has been a persistent issue, with an average of 10\%, peaking above 20\% in 1990 and 2008. The situation improved after 2010. Figure~\ref{fig: sri-gdp-infl} demonstrates the growth and inflation rates for Sri Lanka since 1980.

While Sri Lanka is well-known for its tea exports, the service sector has also become vital. As of 2022, the country's economic sectors are divided as follows: services contribute 56.1\%, industry accounts for 30.3\%, and agriculture makes up 8.7\% of its GDP. Ports and airports play a significant role in the country's emergence as a shipping and aviation hub. The Port of Colombo, in particular, stands as the largest transshipment hub in South Asia.

In the original article where the terminology ``Debt-trap Diplomacy'' was coined, \citet{Chellaney_2017} specifically mentions the predicament faced by the Sri Lanka government. He argues that the China government supported large infrastructure projects in Sri Lanka and provided heavy loans to the government, and as the projects eventually failed to repay the debt, the country became ensnared in the concessions to China. \autoref{fig: sri-lanka-debt-ts} shows the change in composition of the creditors to Sri Lanka.

China started to provide loans to Sri Lanka in 2005, and during 2006 to 2008, the debt amount to China stood around \$1 billion, or roughly 2.9\% of GDP. Starting from 2009, debt to China increased to \$3 billion and reached \$7.5 billion in 2014, accounting for 9.5\% of GDP. The loans are primarily composed of constructions of infrastructures, such as the Hambantota International Port, the Mattala Rajapaksa International Airport, and several road rehabilitation projects.

The issue of the Hambantota Port is regarded as a typical example of a debt-trap~\citep*{Moramudali_2020}.
The construction of the port was initiated in 2007 and entrusted to the state-owned Chinese companies --- China Harbour Engineering Company and Sinohydro Corporation.
The first phase started in 2008, when China lent \$307 million at a yearly interest rate of 6.3\% through the Chinese Exim Bank.\footnote{
    From AidData Project ID \#33409, ``China Exim Bank provides \$306.7 million buyers credit loan for Phase I of Hambantota''}
The second phase started in 2012 shortly after the completion of the first phase, which transfers the Hambantota Port into a container port\cite{}. During the second phase, another \$304 million was provided to Sri Lanka via commercial loans \citep{Horn-Reinhart-Trebesch-21}.


China's involvement in Sri Lanka's infrastructure development was facilitated by President Mahinda Rajapaksa, during which China became Sri Lanka's leading investor and lender. This gave China significant diplomatic leverage over Sri Lanka \citep*{Chellaney_2017}.
However, when Rajapaksa was unexpectedly defeated in the early 2015 election by Maithripala Sirisena, who campaigned on the promise to extricate Sri Lanka from the Chinese debt trap, work on major Chinese projects was suspended.

Sri Lanka's government, however, was already on the brink of default, and Sirisena eventually acquiesced to a series of demands by China in 2017, including the sale of 70\% stake in the Hambantota Port to China Merchants Port (CM Port) and a 99-year lease to China.%
\footnote{
    This narrative originates from \citet*{Chellaney_2017}. \citet*{Brautigam-meme-2020}, however, provides a different narrative. She mentions that ``\emph{The proceeds were used to increase Sri Lanka's
    US dollar reserves in 2017-18 with a view to the repayment of maturing international sovereign bonds \dots Therefore, the sale of Hambantota was originally a fire sale designed to raise money to deal
    with larger debt problems.}''
}
Notably, as argued by \citet*{Moramudali_2019}, the lease did not write off the loans obtained to construct Hambantota Port. The proceeds from the lease were used to boost the country's dollar reserves in 2017-18, especially in preparation for the large amount of external debt that needed to be serviced when international sovereign bonds matured in early 2019. This means that the lease is not a debt-equity-swap, as common narratives elaborated \citep*{Moramudali_2020}.

The Mattala Rajapaksa International Airport appears to have the same problem as the Hambantota Port. The airport was launched in 2009, and \$181 million loan with 2\% interest rate was provided by the Chinese Exim Bank. The airport opened in 2013, but according to the Civil Aviation Authority of Sri Lanka, only 21,000 passengers were served at the airport in 2014. It was criticized as ``the world's emptiest international airport'' \citep*{shepard-16-airport-empty}.

China has also been actively providing loans for expressway and road projects in Sri Lanka. For instance, in 2009, loans amounting to \$1.14 billion were allocated to the Colombo-Katunayake Expressway (CKE). Between 2010 and 2011, road construction loans reached \$1.51 billion, which included various rehabilitation projects for northern roads and highway sections. In 2014 a substantial amount of \$1.99 billion was allocated to the construction and improvement of roads and expressways, contributing significantly to the increase in debt depicted in Figure~\ref{fig: sri-lanka-debt-ts} for the same year.

Sri Lanka eventually declared a suspension of payment on most foreign debt from April 12, 2022. According to the International Debt Statistics (IDS), debt stock of Sri Lanka in 2021 had already reached at least \$56 billion, yielding a 63\% debt-to-GDP ratio, under which debt to China was at least \$7.22 billion.\footnote{
    The debt statistics reported in the International Debt Statistics (IDS) are underestimated as under the Debt Reporting System (DRS), state-owned commercial banks are not included. However, according to the definition of Chinese official lendings adopted in \citet*{Horn-Reinhart-Trebesch-21}, this term should be included. As a result, the debt level in IDS might be underestimated.
}
Whether Sri Lanka was indeed already under the extreme ``brink of default'' during 2015 is a major gap in the literature of sovereign default that has not yet been investigated.
