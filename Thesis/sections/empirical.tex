Typically, a model under the Eaton-Gersovitz framework does not have an analytical solution. Therefore, the optimal default set defined by \refeq{eq:default-set}, as well as the value functions and the policy functions, must be obtained numerically via the technique of value function iteration.
This requires the assignment of functional forms as well as structural parameters that matches the economy.
I follow the functional forms and the calibration approach introduced in \citet{Na-18} and \citet{Hinrichsen_2020-chapter4}.

\section{Calibration}
\label{sec: calibration}

\subsection{Functional Forms}
Following \citet{Na-18}, the time unit is assumed to be one quarter, and the periodic utility function is assumed to be the constant relative risk aversion (CRRA) type
\begin{equation}
    \label{eq:CRRA-utility}
    U(c_t) = \frac{c_t^{1-\sigma} - 1}{1 - \sigma},
\end{equation}
where $\sigma$ is the inverse of elasticity of intertemporal substitution of the consumption.
The aggregator function for tradable and non-tradable consumption takes the constant elasticity of substitution (CES) form
\begin{equation}
    \label{eq:aggregator-function}
    c_t = A(c^T_t, c^N_t) =
        \left[ a \left( c^T_t \right)^{1- \frac{1}{\xi}} +
            (1 - a) \left( c^N_t \right)^{1- \frac{1}{\xi}}
        \right]^{\frac{1}{1 - \frac{1}{\xi}}}.
\end{equation}
The CES aggregator states that the share of tradable consumption is $a \in [0,1]$, and the elasticity of substitution between the tradable and non-tradable consumption is $\xi$.
Moreover, following the literature, to make the consumption of tradable goods $c^T_t$ and the external debt $d_t$ independent of the outputs in the nontradable sector in the equilibrium,
assume that the inter- and intratemporal elasticity of substitution is equivalent \citep*[See][Chapter 9.5]{Uribe-Schmitt-Grohe-textbook}.
That is,
\begin{equation}
    \label{eq:xi-sigma}
    \xi = \frac{1}{\sigma}.
\end{equation}
The production technology for the nontradable goods follows a simple form
\begin{equation}
    \label{eq:production-function}
    y^N_t = F(h_t) = h_t ^\alpha.
\end{equation}
The loss-function in \refeq{eq:ytt} is positive and increasing with $y^T_t$, and following \citet{Chatterjee-12}, I adopt the quadratic form with two parameters
\begin{equation}
    L(y^T_t) = \max \left\{
        0, \delta_1 y^T_t + \delta_2 \left( y^T_t \right)^2
     \right\}.
\end{equation}
This is also adopted in \citet{Na-18}. In this setting, if we set $\delta_1 < 0$ and $\delta_2 >0$, the output-loss increases as $y^T_t$ increases, indicating that the more a country is endowed, the more it loses during default.

\subsection{Calibration of Sri Lanka}
The model is calibrated to Sri Lanka before 2008, when the Chinese government started to provide the increasing amount of loans.
China started to provide loans to Sri Lanka in 2005, and during 2006 to 2008, the debt amount to China remains to be around \$1 billion, which is roughly 2.9\% of GDP. Starting from 2009, debt to China has increased to \$3 billion, and it reached \$7.5 billion in 2014, accounting for 9.5\% of GDP
(see \autoref{fig: sri-lanka-debt-ts}).

I proxy the output process of \refeq{eq:ar1-output} by the detrended log-real-GDP of Sri Lanka from 1980 to 2021. Considering that the seasonality in the quarterly data for Sri Lanka might impose a higher volatility estimated in the AR(1) process, I follow \citet{Hinrichsen_2020-chapter4} and estimate the annual data over 1980 to 2021. I obtain the cyclical component of the output by filtering the time series with an HP-filter with smoothing parameter $\lambda$ set to 100.
Estimation of the AR(1) on the cyclical component thus yields $\rho = $ 0.9114 and $\sigma_u = $ 0.0123\footnotemark{}. \autoref{fig:decompose-gdp} presents the decomposition of the log-real-GDP.
\footnotetext{Since the AR(1) estimation is conducted on annual data, the estimated coefficients must be quarterized. Specifically,
$\rho = 1 - \frac{1 - \hat{\rho}}{4}$, and $\sigma_u = \frac{\hat{\sigma}}{\sqrt{4}}$, where $\hat{\rho}$ and $\hat{\sigma}$ are the estimated parameters for the AR(1) via OLS.}

The global risk-free world interest rate $r^*$ is set to match the U.S. 3-month treasury bill rate during 1990 to 2007\footnote{
    Quarterly averaged data retrieved from FRED. 1980s are excluded due to the fact that the FED sets it rate extremely high to fight the inflation, and 2008 is excluded due to sudden drop caused by the financial crisis. The average of 3-month treasury bill rate over 1990 to 2007 is 4.10\%.
}, which is roughly 4\% annually, or 1\% for one quarter. This is in line with \citet{Chatterjee-12}.
The probability of reentry is difficult to assess since Sri Lanka encountered its first default in April 12, 2022, and is not yet undergoing the process of restructuring. As a result, following \citet*{Chatterjee-12} and \citet*{Hinrichsen_2020-chapter4}, I set the probability of reentry to 0.0385, which implies that the country will be in default on average for about 6.5 years.

The labor share is set as $\alpha=$ 0.65 based on the calibration in \citet*{Jegajeevan-Sri-Lanka-DSGE}, which matches the estimation of labor share in \citet{duma2007sri}. The share of tradable consumption is approximated by the share of tradable output in total output, suggested in \citet{Uribe-Schmitt-Grohe-textbook}. Calculating the mean of tradable-to-GDP ratio over 1980 to 2021, I set the value as $a =$ 0.36. The elasticity of substitution between tradable and nontradable goods $\xi$ follows the cross-country estimation by \citet*{Stockman-Tesar-95}, which is 0.44, and is approximated to 0.5. According to the assumption in \refeq{eq:xi-sigma}, $\sigma=1/\xi=$ 2. This is inline with the calibration for most real-business-cycles \citep{Uribe-Schmitt-Grohe-textbook,Na-18}.

Following \citet{Na-18}, the rest of the parameters $\left( \beta, \delta_1, \delta_2 \right)$, which is respectively the subjective discount factor and the two parameter for the loss function, is chosen to match three equilibrium outcomes\footnote{
    In particular, I use the grid search algorithm to search for the optimal values for the three parameters. Essentially, VFI must be proceeded for each triplet of the parameters to obtain the targeting equilibrium outcomes. Details are mentioned in \autoref{sec:computation}.
}:
\begin{enumerate*}[label = (\roman*)]
    \item the average debt-to-GDP ratio in periods of good standing is 65\% per quarter;
    \item the frequency of default is 1.37 times per century; and
    \item the average output loss is 10.5\% per year conditional on being in financial autarky.
\end{enumerate*}
The average debt-to-GDP ratio to be targeted is motivated by the fact that the average annual debt-to-GDP ratio in the data is about 44\%.
The value is calculated by averaging the nominal debt-to-GDP ratio over 2001 to 2008\footnotemark{}.
\footnotetext{Data source: International Debt Statistics. The period of year is chosen to be 8 years before China's increasing support of loans. The time span of 8 years is inline with that in \citet{Uribe-Schmitt-Grohe-textbook}.}
Multiplying this by an average of 37\% haircut\footnote{The average sovereign haircut between 1970 and 2010 \citep{Cruces-Trebesch-13}.} implies that about 16.28\% of the debt is unsecured annually\footnotemark{}.
\footnotetext{In the mode, we assume that the country defaults on 100\% of the debt, hence this approach is necessary to handle the case of a haircut.}
Since we are dealing with a model with quarterly period, this results in the 65\% debt-to-GDP ratio targeted during calibration.
The average frequency of default is justified by the fact that Sri Lanka experienced its first default on 2022 since its independence in 1948; this give an average of 1.37 times per century\footnote{$\frac{1}{2022-1948} \times 100 \approx 1.37$}. Finally, (?)
% following the default of Sri Lanka since April 2022, the GDP growth is -7.4\%, -11.5\%, and -12.4\%, for the subsequent three quarters\footnote{Data Source: Central Bank of Sri Lanka}. This implies a geometric mean of -10.5\%.
Table \ref{tab:cal-sri-lanka} summarizes the calibrated parameters and their sources.



\subsection{Calibration of Pakistan}
The calibration strategy for Pakistan is similar to that of Sri Lanka.
The parameters for the output process is obtained from the cyclical component of the HP-filter on the annual log-real-GDP for Pakistan from 1980 to 2021, which yields $\rho = $ 0.9008 and $\sigma_u=$ 0.0111 (see \autoref{fig:decompose-gdp}).
The risk-free interest rate remains to be 4\% annually, hence $r= $1\%.
Pakistan defaulted on January 1999, completed its debt restructuring on December 1999 \citep{SPGlobal-default-report}, but gained partial reaccess (flows > 0) in 2004, and full reaccess (flow > 1\% of GDP) in 2006 \citep*[][Table 5.6]{trebesch-2011-sovereign}.
The model adopted in my thesis does not distinguish between partial or full reaccess, hence the reentry period is set as the first year Pakistan gain positive flow of debt. Accordingly, the reentry period is set to 6 years (24 quarters), and $\theta=$ 0.0417.

The labor share is set as 0.4 to match the capital share in real GDP, following \citet{Pakistan-DSGE-calibration}. The share of tradable consumption is calibrated according to the tradable-to-GDP ratio over 1980 to 2021, which gives $a=$ 0.33. The intratemporal elasticity of substitution of consumption $\xi=$ 0.5 and $\sigma=$ 2, following the same justification for Sri Lanka \citep{Pakistan-DSGE-calibration,Uribe-Schmitt-Grohe-textbook}.
Nominal wage rigidity is set as $\gamma=$ 1.048 based on the empirical estimation of downward wage rigidity in 2014 by \citet*{wage-rigidity-data}.

Finally, the triplet $\left( \beta, \delta_1, \delta_2 \right)$ is also chosen to match the three equilibrium results:
\begin{enumerate*}[label = (\roman*)]
    \item the average debt-to-traded-GDP ratio in periods of good standing is 18\% per quarter;
    \item the frequency of default is 4 times per century; and
    \item the average output loss is 7\% per year conditional on being in financial autarky.
\end{enumerate*}
\begin{enumerate}[label = (\roman*)]
    \item
    The average debt-to-GDP-ratio between 2006 and 2013\footnote{
    Following the same logic as with Sri Lanka, I choose an 8-years window before the increasing amount of loans from China in 2013.}
    is 30\% according to the International Debt Statistics. Multiplying it with a 15\% haircut ratio estimated in \citet{Cruces-Trebesch-13} yields a 4.5\% of annual unsecured debt, which gives an 18\% unsecured debt-to-GDP ratio quarterly.
    \item History defaults or restructurings in Pakistan according to the BoC-BoE Sovereign Default Database states that Pakistan default (or rescheduled) to IMF 2 times (1978, 1980), to Paris Club 5 times (1972, 1974, 1981, 1999, 2001)\footnote{
        These defaults differ. In 1972, 1974 and 2001, Pakistan was dealt with under \emph{ad hoc terms}, meaning that the creditors is able to fix the terms and conditions for restructuring or rescheduling; in 1981, Pakistan was treated under the \emph{Classic terms}, in which it received a waiver on interest; in 1999, which is the default episode most widely recorded, Pakistan was dealt with under the \emph{Houston terms}, which offers an even longer grace period \citep{pakistan-default-start}.
    }, and to China 2 times (2002, 2021). This yields an average frequency of 4 times per century per creditor\footnote{$\frac{2+5+2}{2022 - 1947}\times 100 \times \frac{1}{3}$}.
    \item
    Following calculations based on \citet*{zarazaga-12}, the capital-output ratio dropped from 1.43 to 1.35 after the default episode. This yields an output cost of 5\% each year if the loss is attributed solely on the default in 1999. See Appendix \ref{ap: zarazaga} for more detail.
\end{enumerate}
Table \ref{tab:cal-pakistan} summarizes the calibrated parameters for Pakistan.


\section{Numerical Computation}
\label{sec:computation}
The approximated equilibrium is obtained by conducting value function iteration over an $n_y \times n_d$ discretized and equally spaced state space,
where $n_y = $ 200 is the number of grids for the output process and $n_d=$ 200 is the number of grids for the debt \citep{Na-18}. Denote $[\underline{y}^T, \overline{y}^T]$ as the lower and upper bound of output grid. Following \citet{Uribe-Schmitt-Grohe-textbook}, this is set as $[-4.2 \sigma_u, 4.2 \sigma_u]$. Also following the authors, since the average debt levels for both countries do not exceed 150\%, the upper bound for debt is set as 1.5, therefore the debt range for the value function iteration is $[\underline{d}, \overline{d}]=[0,1.5]$.

I follow \citet{Schmitt-Uribe-16} and discretize the AR(1) process for output by constructing a transition probability matrix over the grids of the output.
A time series of 10 million observations was generated based on \refeq{eq:ar1-output}. Each observation is then assigned to the nearest grid point among the 200 discrete values of $\ln y^T$. The discretized series is analyzed to calculate the probabilities of transitioning from one discrete state to another in consecutive periods.
To obtain the transition probability matrix, a $200\times200$ matrix is initialized with zeros. For each pair of consecutive observations, the corresponding element in the matrix is incremented by 1. After considering all the observations, the matrix is normalized by dividing each row by the sum of its elements. This results in the estimated transition probability matrix, which effectively captures the covariance matrices of order 0 and 1
\citep*{Uribe-Schmitt-Grohe-textbook}.

Value function iteration is then conducted for a given set of calibrated parameters, which is composed of the set of fixed parameters $(\rho, \sigma_u, r^*, \theta, \alpha, a, \xi, \gamma)$ and the tuple $(\beta, \delta_1, \delta_2)$. The default dynamic and moments can then be conducted once the value function converges. Following \citet{Uribe-Schmitt-Grohe-textbook} and \citet{Na-18}, a simulation of the model based on the policy function is conducted sequentially\footnote{
    The main randomness comes from the AR(1) output. Once $y^T_{t+1}$ is determined, along with determination of the debt level for next period $d_{t+1}$, we have the state $(y^T_{t+1}, d_{t+1})$ for $t+1$, which is a grid-point. Values for the rest of the endogenous variables such as $c_{t+1}, I_{t+1}$ can then be obtained by extracting the corresponding value from the state grid-point on the policy function.
} for 1.1 million iterations, of which the first 0.1 million periods are discarded.

A critical procedure of the calibration is to select the tuple $(\beta, \delta_1, \delta_2)$ such that it matches the three targets described in section \ref{sec: calibration}, which are the default frequency, debt-to-GDP ratio, and output loss.
In particular, the tuple is chosen to minimize the Euclidean distance between the target value and the corresponding moments generated by the dynamic simulation\footnote{
    Default frequency is evaluated by taking the mean of $\left\{ I_{t} \right\}_{t \in T}$ and multiply it by $100\times 4$ quarters per century. Debt-to-GDP ratio is calculated by taking the mean of $\left\{ d_t / y^T_t \right\}_{t \in \text{Good T}}$, which is the average debt-to-tradable-output ratio condition on being in good standing. Output loss is evaluated by taking the mean of $\left\{ (y^T_t - \tilde{y}^T_t)/y^T_t \right\}_{t \in \text{Bad T}}$, which is the average output difference between the endowment and the actual received output, divided by the endowment, over periods of bad standings.
}. Since the value function iteration and dynamic simulation are time-consuming processes, I use the surrogate optimization algorithm in MATLAB as the searching algorithm, as it is suitable for minimizing time-consuming objective functions.

The estimated results are reported in Table~\ref{tab: calibration-compare}.
The upper part of the table presents the estimated values of $(\beta, \delta_1, \delta_2)$, and the lower part of the table shows the difference between the target and simulated moments under the optimal calibration (see the rows where the first column is ``HP'').
There exist errors in the minimal distance which is inevitable upon the computation. For Sri Lanka, the debt-to-tradable-GDP ratio is 1.73, lower than the target by 0.02, and the output loss is higher by 3\%. Frequency of default per century gives 1.26, which is about half the frequency I target. Similar results are obtained in the case of Pakistan. The debt-to-tradable-GDP ratio fits pretty well, and the output loss is lower by 1.3\%, but the frequency of default is also 1.26 times per century, also about have of the targeted default frequency.\footnote{
    With that being said, the results are actually pretty satisfying. The target of defaulting 2.6 times per century for both countries is set due to the ambiguity of determining default episodes, so I target a benchmark value of 2.6 times. However, Sri Lanka officially declared default for the first time in 2022, which yields $1/(2023-1948) \times 100 \approx 1.3$ times per century. As for Pakistan, \citet{Uribe-Schmitt-Grohe-textbook} and \citet{SPGlobal-default-report} only record the default event in 1999, which up to now yields $1/(2023-1947) \times 100 \approx 1.3$ times per century. These calculations are naive and inexact, but are roughly able to justify the estimated value of frequency of 1.26. A more robust target requires longer historical data throughout the history, which I am unable to tackle in this thesis.
}

\section{Default Set}
As described in Section \ref{sec: model-default-decision} and Appendix \ref{ap: default-set}, the default decision is jointly decided by the current debt level and its output for tradable goods during the period. Under a certain debt stock $d_t$, the less the tradable output $y^T_t$ it is endowed, the lower the benefit a country receives when it chooses to repay its debt.
Ultimately, if the output falls below a certain threshold value, it becomes optimal for the country to default.
\citet{Hinrichsen_2020-chapter4} compares the theoretical scenarios of defaulting with the actual data with a graphical approach. He maps the default set evaluated via value function iteration into the state space $(y^T_t, d_t)$, with the output as the horizontal axis and the debt level as the vertical axis. A grid on the space is colored in gray if the corresponding debt-output pair is \emph{not} in the default set. Figure~\ref{fig:ds} demonstrates the default set for Sri Lanka and Pakistan. The debt level in the default set plot differs between the two countries due to the difference in the calibration and the targeted equilibrium moments. As shown in the figure and proven in Appendix \ref{ap: default-set}, the upper limit of a default set (the intersection of the white and gray areas given a certain debt level) increases as the debt level increases, indicating that it requires a higher endowment of tradable output in order to incentivize the country to continue to fulfill its obligations to repay.\footnote{
    In the default set depicted in \citet{Hinrichsen_2020-chapter4}, there is a discontinuity in the upper bound of the default set. This indicates that when the output level drops to a certain threshold, the country chooses to default, regardless of the current debt level. This is contradicted to the analytical conclusion proved in Proposition \ref{prop3} in Appendix \ref{ap: default-set}. During a private correspondence, the author acknowledged this mistake in his doctoral thesis, but he has already corrected it in his latest book.
}

The actual data is then plotted on the default set for comparison. The output level corresponds to the cyclical component of the per capita tradable-GDP obtained by the HP-filter for the respective year.
The corresponding debt level is calculated by multiplying the debt-to-nominal-tradable-GDP ratio by the haircut ratio and 4 quarters. This calculation is necessary because the targeted debt level during calibration specifically refers to the quarterly unsecured portion of the debt, and thus the data should undergo the same calculation to accurately represent the unsecured debt component.

Furthermore, the debt data used in this analysis consists of two components: the debt stock excluding China obtained from the International Debt Statistics (IDS), and the debt stock specifically for China sourced from the database created by \citet*{Horn-Reinhart-Trebesch-21}. It is worth noting that Chinese government loans channeled through state-owned enterprises are typically not reported in the World Bank Debt Reporting System (DRS), which is the basis for IDS. This omission leads to an underestimation of the actual debt burden faced by the country. The additional loans provided by state-owned commercial banks, referred to as ``hidden debts'' by \citet*{Horn-Reinhart-Trebesch-21}, are included in their database. Consequently, in order to capture the total debt burden accurately, the total debt stock reported in IDS is adjusted by excluding the debt to China from IDS and adding the debt to China obtained from the database by \citet*{Horn-Reinhart-Trebesch-21}.

\subsection{Default Decision for Sri Lanka}

Figure \ref{fig: ds-sri-data-with-china} plots the debt-output pair for Sri Lanka from 2007 to 2017. Each point represents the actual data pair for the year, adjusted by the haircut ratio. Recall that China initiated its constructions in infrastructures such as the Hambantota International Port in 2007, and Mattala Rajapaksa International Airport in 2009. The selection of this time span allows us to examine the change in debt burden for Sri Lanka comprehensively in a chronological way.

As shown in the figure, at the initial phase of China's investment in 2007 and 2008, Sri Lanka is still under its safe zone of not defaulting, despite that it experienced a little drop in the output. Sri Lanka first landed into the default set during 2009, as debt to China reached \$3 billion. During 2010 and 2011, Sri Lanka ended its 25-years civil war, and experienced an unprecedented output growth. Agriculture sector in 2010 grew for 7\%, and industry sector grew for 8\%. This rapid increasing in GDP largely lowered the debt-to-tradable-GDP ratio, both in 2010 and 2011, and hence Sri Lanka was able to avoid the risk of default and stayed under the non-default set. However, as GDP growth in tradable good slowed down to a rate of 5\%, the debt to China increased by 6\%, which slowly pushed Sri Lanka to the brink of default. Eventually, in 2013, as total debt reached \$41 billion, in which China accounted for 13\%, Sri Lanka was again entering the default set. The status continued until 2016, with 2014 being an ambiguous exception.

It is worth mentioning that in 2015, as President Maithripala Sirisena defeated former President Mahinda Rajapaksa in the election, he promised to extricate Sri Lanka from the debt burden from China. However, according to Figure \ref{fig: ds-sri-data-with-china}, Sri Lanka was too indebted during 2015 and 2016, and it would be difficult for the new government to take actions besides undergoing debt restructuring. Eventually, in 2017, the infamous event of Sri Lanka leasing the Hambantota International Port for 99 years and selling 70\% of the stake to China Merchants Port stroke the headline. This action, however, did not write off the loans for the construction of the infrastructures, as it is used as a raise in money to repay debt to other official creditors \citep{Brautigam-meme-2020, Moramudali_2019}. Regardless of the narrative of whether the lease is involuntary or not, along with the outstanding output level in 2017, Sri Lanka was back to the non-default set after the lease.
Observe that all the cyclical components of the tradable goods (the x coordinate) are above 1 after 2011. This indicates that Sri Lanka is under the default set even when the output is mostly above the steady state level of tradable outputs.

Sri Lanka did not have issues on defaulting before 2007. Debt to China indeed boosted Sri Lanka into the predicament of debt unsustainability, or in other words, pushed Sri Lanka into a debt trap. I refer to this as the ``type-one debt trap.'' Formally defining, the ``type-one debt trap'' refers to a scenario where a financially stable country is pushed into a state of debt unsustainability, leading to a high risk of default on its debt obligations. This debt trap occurs as a result of external interventions, particularly an influx of excessive loans, which disrupt the financial equilibrium of the country.

An intriguing question arises: What if the debt to China is not included, holding other conditions fixed? \citet{Hinrichsen_2020-chapter4} compares the debt for countries in war reparations with and without indemnities, and here I adopt the same approach. Figure \ref{fig: ds-sri-data-x-china} shows the comparison of the data. In the figure, the solid black dots represent data points including debts owed to China, while the gray dots represent data points excluding debts from China.
It is obvious that, \emph{ceteris paribus}, all years spanning from 2009 to 2017 are not under the default set. This is, indeed, an inexact comparison.
First, the debt level given in the model is endogenous, and we do not distinguish between China and other lenders. It is possible that Sri Lanka sought for other creditors for loans if the infrastructures were planned voluntarily, as mentioned in \citet{Brautigam-meme-2020}. In this case the debt level in the absence of China can be underestimated.
Second, the impact on GDP would be uncertain if China had not been involved in the construction of the infrastructure projects. For instance, \citet{Bandiera-Vasileios-BRI-debt} estimates from a long-term growth model that in 2020, the additional growth from investment on the Belt and Road Initiative (BRI) is about 0.08\%.
Estimations of the extra growth due to investments during 2009 to 2017 are crucial in order to examine the counterfactual debt burden in the absence of China's intervention. If investments in China indeed contributed to the GDP growth during the periods, the debt-to-tradable-GDP ratio excluding China is also underestimated.

\subsection{Default Decision for Pakistan}

Figure \ref{fig: ds-pak-data-with-china} presents the debt-output pair for Pakistan spanning from 2010 to 2017. The initiation of extensive infrastructure projects under the China-Pakistan Economic Corridor (CPEC) took place in 2015, marking a significant milestone. While the focus of comparison primarily revolves around the two years preceding China's intervention, including the data from 2010 allows for a comprehensive understanding of Pakistan's debt situation.

Pakistan encountered the most severe flood in 2010, which caused an estimated \$9.7 billion loss in infrastructure. The devastation brought Pakistan to the default set in 2010.
However, since the disasters in 2010 and 2011, Pakistan was not under the default set during 2013. Is relatively safe from being in the default status, as even an 8\% contraction in the output (if $y^T_t$ = 0.92) is not making Pakistan willing to default. Debt level started to increase intensively, and up to 2015, debt to China had reach about \$15 billion, surpassing the amount of debt to World Bank, which in that time was about \$13 billion. Since then, China has become the largest creditor to Pakistan (see Figure~\ref{fig: pakistan-debt-ts}). As a result, debt-to-tradable GDP increased, and both data points in 2014 and 2015 located above 2013 in Figure~\ref{fig: pak_d_t_trad}. In 2016, 6 additional major loans on infrastructure constructions mainly on the energy sector was launched, which summed up to \$8.6 billion, and hence in 2016 Pakistan was around the brink of default. Details about the projects are described in Section \ref{sec: intro-pak}.
Eventually in 2017, with 3 more major loans in power project initialized, the total debt stock is up to \$108 billion, and compared with the nominal tradable GDP of \$136 billion, the debt-to-tradable-GDP is 80\%, which corresponds to about 120\% unsecured debt per quarter. This pushed Pakistan into the default set, and made Pakistan strangled in the debt trap.

The provision of a large number of power projects to Pakistan by China is primarily aimed at addressing its most urgent needs in power supply. Pakistan has been suffering from energy shortage since the 1990s. During 2010 to 2013, the former Pakistan Prime Minister Yousuf Raza Gilani launched an ``energy policy'' that aimed to resolve the power crisis, with measures including banning neon signs, closing street markets early, and introducing of 13 independent power producers (IPP).\footnote{
    BBC ``Pakistan's PM announces energy policy to tackle crisis,'' April 22, 2010
}
While the construction of large-scale power plants, including nuclear plants, could have directly increased energy supply, the Pakistani government primarily focused on energy conservation policies. Financial restrictions may have been the main concern for the government that prevented them from initializing large constructions, especially after the tragic natural disasters such as the 2010 floods in Pakistan.\footnote{
    Asian Development Bank and World Bank jointly estimated the total cost of the damage at approximately \$10.1 billion, and the reconstruction costs at least \$6.8 billion \citep{pakistan-flood}.
}

China grasped this opportunity, and provide excessive aids when Pakistan is under this turmoil, thereby further exacerbating the already onerous debt obligation. I refer to this as the ``type-two debt trap.''
This type of debt trap arises when a financially distressed country is already insolvent or facing severe financial difficulties, making it challenging to secure additional funds from other countries or lenders.\footnote{
    The Moody's rated Pakistan as Caa2 in 2012. Data from Trading Economics.
}
However, a specific entity or source steps in and provides substantial loans to the distressed country, exceeding what would typically be considered reasonable or sustainable levels. Despite the overwhelming nature of these loans, the distressed country, in its desperate need for financing and infrastructures, accepts them. This sets the stage for a desperation-driven debt trap, wherein the distressed country becomes increasingly reliant on the lending entity for funds, exacerbating its financial vulnerability and restricting its options for recovery.

Similar to Sri Lanka, the following attempts to illustrate the economy of Pakistan excluding China. Figure~\ref{fig: ds-pak-data-x-china} illustrates the debt-output relationship when excluding China's interventions. Throughout the entire time span, the debt-to-tradable-GDP ratio remains well below the default threshold. An exception is the year 2010, when the flood stroke Pakistan, as it would default regradless of the debt to China. Notably, in 2017, the ratio closely resembles that of 2013. However, it is important to reiterate that this result does not consider the potential increase in debt owed to other creditors or the benefits derived from investments related to the BRI.

As mentioned previously, Pakistan is craving for solutions for its power supply, but is not under the financial status of realizing it. It is reasonable to argue that without the loans and projects initiated by China, Pakistan would not have undertaken these energy projects, and its debt would not have accumulated dramatically from other official creditors.

Another aspect of uncertainty when assessing the counterfactual debt-to-tradable-GDP ratio is the impact on GDP following China's investment in infrastructure. According to estimates from \citet*{Bandiera-Vasileios-BRI-debt}, using a long-term growth model, additional investment from the BRI accounted for 1.93\% of GDP, resulting in a 0.41\% increase in GDP compared to the baseline model in 2020. This effect is five times larger than that observed in Sri Lanka. Although estimations for the period between 2013 and 2017 have not been conducted, a preliminary observation can be made from Figure \ref{fig: pak_d_t_trad}. The nominal tradable GDP demonstrates increasing growth after 2013, despite the total debt level increases at a faster pace. Since this observation is not sufficient for causal inference, further examination of the infrastructure's contribution to GDP is a topic worthy of future research.

\section{Robustness Check}
The value function iteration of the model depends on the persistency and volatility of the AR(1) process for the tradable output, namely $(\rho, \sigma_u)$. The parameters are obtained with the cyclical component after conducting the HP-filter. In addition, the x-coordinate of the data points plotted on the default set is also obtained with the cyclical component, representing the deviation from the trend. One may concern that the method of detrending might yield a different conclusion. In \citet{Na-18} and \cite{Hinrichsen_2020-chapter4}, the output process is detrended using the log-quadratic filter.\footnote{
    In specific, assume that the real GDP can be expressed by the cyclical component and trend (secular) component $y_t = y^s_t + y^c_t$. The components are estimated by running the OLS: $y_t = a + bt + ct^2 + \epsilon_t$, and then set $y^c_t = \epsilon_t$ and $y^s_t = a + bt + ct^2$ \citep{Uribe-Schmitt-Grohe-textbook}.
}
Generally, the volatility for the cyclical component obtained with log-quadratic filter is higher than that of HP-filter \citep{Uribe-Schmitt-Grohe-textbook}.

Figure~\ref{fig:decompose-trad-logq} illustrates the decomposition of the (per capita) tradable output using the log-quadratic filter. It yields $(\rho, \sigma_u)=$ (0.9325, 0.0266) for Sri Lanka and $(\rho, \sigma_u)=$ (0.9239, 0.0174) for Pakistan. Compared to the unconditional standard deviation of 4.37\% for Sri Lanka's per capita tradable output using the HP-filter, the log-quadratic filter yields 7.38\%; similarly, the unconditional standard deviation for Pakistan's per capita tradable output using the HP-filter is 2.21\%, while the log-quadratic filter yields 4.55\%.
See Table~\ref{tab:ar1-filter} for the comparison.

The default set and the corresponding debt-output pairs when the output process is obtained via log-quadratic filter is shown in Figure~\ref{fig: ds-logq}. Interestingly, the number of years when Sri Lanka was within the default set reduces, as depicted in Figure~\ref{fig: ds-sri-logq}.
In contrast to the results obtained with the HP-filter, the year 2009 is now considered safe, albeit on the verge of default. The remaining two years within the default set are 2015 and 2016, coinciding with the period when the newly elected president expressed his determination to alleviate Sri Lanka's burden of debt to China. These findings regarding the predicament faced by the new government remains robust when using the log-quadratic filter.
As for the case of Pakistan, the year 2017 is no longer under the default set, contrary to the result obtained with the HP-filter. Nevertheless, the trend of the debt-to-tradable-ratio kept increasing, and during the year 2017, the output deviated towards the positive direction ($y^T_t > 1$). If the output shock were to contract just by a mere 1\% (at $y^T_t = 0.99$), Pakistan would be falling into the default set.