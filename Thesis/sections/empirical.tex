Typically, a model under the Eaton-Gersovitz framework does not have an analytical solution. Therefore, the optimal default set defined by \refeq{eq:default-set}, as well as the value functions and the policy functions, must be obtained numerically via the technique of value function iteration.
This requires the assignment of functional forms as well as structural parameters that matches the economy.
I follow the functional forms and the calibration approach introduced in \citet{Na-18} and \citet{Hinrichsen_2020-chapter4}.

\section{Calibration}
\label{sec: calibration}

\subsection*{Functional Forms}
Following \citet{Na-18}, the time unit is assumed to be one quarter, and the periodic utility function is assumed to be the constant relative risk aversion (CRRA) type
\begin{equation}
    \label{eq:CRRA-utility}
    U(c_t) = \frac{c_t^{1-\sigma} - 1}{1 - \sigma},
\end{equation}
where $\sigma$ is the inverse of elasticity of intertemporal substitution of the consumption.
The aggregator function for tradable and non-tradable consumption takes the constant elasticity of substitution (CES) form
\begin{equation}
    \label{eq:aggregator-function}
    c_t = A(c^T_t, c^N_t) =
        \left[ a \left( c^T_t \right)^{1- \frac{1}{\xi}} +
            (1 - a) \left( c^N_t \right)^{1- \frac{1}{\xi}}
        \right]^{\frac{1}{1 - \frac{1}{\xi}}}.
\end{equation}
The CES aggregator states that the share of tradable consumption is $a \in [0,1]$, and the elasticity of substitution between the tradable and non-tradable consumption is $\xi$.
Moreover, following the literature, to make the consumption of tradable goods $c^T_t$ and the external debt $d_t$ independent of the outputs in the nontradable sector in the equilibrium,
assume that the inter- and intratemporal elasticity of substitution is equivalent \citep*[See][Chapter 9.5]{Uribe-Schmitt-Grohe-textbook}.
That is,
\begin{equation}
    \label{eq:xi-sigma}
    \xi = \frac{1}{\sigma}.
\end{equation}
The production technology for the nontradable goods follows a simple form
\begin{equation}
    \label{eq:production-function}
    y^N_t = F(h_t) = h_t ^\alpha.
\end{equation}
The loss-function in \refeq{eq:ytt} is positive and increasing with $y^T_t$, and following \citet{Chatterjee-12}, I adopt the quadratic form with two parameters
\begin{equation}
    L(y^T_t) = \max \left\{
        0, \delta_1 y^T_t + \delta_2 \left( y^T_t \right)^2
     \right\}.
\end{equation}
This is also adopted in \citet{Na-18}. In this setting, if we set $\delta_1 < 0$ and $\delta_2 >0$, the output-loss increases as $y^T_t$ increases, indicating that the more a country is endowed, the more it loses during default.

\subsection*{Calibration of Sri Lanka}
The model is calibrated to Sri Lanka before 2008, when the Chinese government started to provide the increasing amount of loans.
China started to provide loans to Sri Lanka in 2005, and during 2006 to 2008, the debt amount to China remains to be around \$1 billion, which is roughly 2.9\% of GDP. Starting from 2009, debt to China has increased to \$3 billion, and it reached \$7.5 billion in 2014, accounting for 9.5\% of GDP
(see \autoref{fig: sri-lanka-debt-ts}).

I proxy the output process of \refeq{eq:ar1-output} by the detrended log-real-GDP of Sri Lanka from 1980 to 2021. Considering that the seasonality in the quarterly data for Sri Lanka might impose a higher volatility estimated in the AR(1) process, I follow \citet{Hinrichsen_2020-chapter4} and estimate the annual data over 1980 to 2021. I obtain the cyclical component of the output by filtering the time series with an HP-filter with smoothing parameter $\lambda$ set to 100.
Estimation of the AR(1) on the cyclical component thus yields $\rho = $ 0.9114 and $\sigma_u = $ 0.0123\footnotemark{}. \autoref{fig:decompose-gdp} presents the decomposition of the log-real-GDP.
\footnotetext{Since the AR(1) estimation is conducted on annual data, the estimated coefficients must be quarterized. Specifically,
$\rho = 1 - \frac{1 - \hat{\rho}}{4}$, and $\sigma_u = \frac{\hat{\sigma}}{\sqrt{4}}$, where $\hat{\rho}$ and $\hat{\sigma}$ are the estimated parameters for the AR(1) via OLS.}

The global risk-free world interest rate $r^*$ is set to match the U.S. 3-month treasury bill rate during 1990 to 2007\footnote{
    Quarterly averaged data retrieved from FRED. 1980s are excluded due to the fact that the FED sets it rate extremely high to fight the inflation, and 2008 is excluded due to sudden drop caused by the financial crisis. The average of 3-month treasury bill rate over 1990 to 2007 is 4.10\%.
}, which is roughly 4\% annually, or 1\% for one quarter. This is in line with \citet{Chatterjee-12}.
The probability of reentry is difficult to assess since Sri Lanka encountered its first default in April 12, 2022, and is not yet undergoing the process of restructuring. As a result, following \citet*{Chatterjee-12} and \citet*{Hinrichsen_2020-chapter4}, I set the probability of reentry to 0.0385, which implies that the country will be in default on average for about 6.5 years.

The labor share is set as $\alpha=$ 0.65 based on the calibration in \citet*{Jegajeevan-Sri-Lanka-DSGE}, which matches the estimation of labor share in \citet{duma2007sri}. The share of tradable consumption is approximated by the share of tradable output in total output, suggested in \citet{Uribe-Schmitt-Grohe-textbook}. Calculating the mean of tradable-to-GDP ratio over 1980 to 2021, I set the value as $a =$ 0.36. The elasticity of substitution between tradable and nontradable goods $\xi$ follows the cross-country estimation by \citet*{Stockman-Tesar-95}, which is 0.44, and is approximated to 0.5. According to the assumption in \refeq{eq:xi-sigma}, $\sigma=1/\xi=$ 2. This is inline with the calibration for most real-business-cycles \citep{Uribe-Schmitt-Grohe-textbook,Na-18}.

Following \citet{Na-18}, the rest of the parameters $\left( \beta, \delta_1, \delta_2 \right)$, which is respectively the subjective discount factor and the two parameter for the loss function, is chosen to match three equilibrium outcomes\footnote{
    In particular, I use the grid search algorithm to search for the optimal values for the three parameters. Essentially, VFI must be proceeded for each triplet of the parameters to obtain the targeting equilibrium outcomes. Details are mentioned in \autoref{sec:computation}.
}:
\begin{enumerate*}[label = (\roman*)]
    \item the average debt-to-GDP ratio in periods of good standing is 65\% per quarter;
    \item the frequency of default is 1.37 times per century; and
    \item the average output loss is 10.5\% per year conditional on being in financial autarky.
\end{enumerate*}
The average debt-to-GDP ratio to be targeted is motivated by the fact that the average annual debt-to-GDP ratio in the data is about 44\%.
The value is calculated by averaging the nominal debt-to-GDP ratio over 2001 to 2008\footnotemark{}.
\footnotetext{Data source: International Debt Statistics. The period of year is chosen to be 8 years before China's increasing support of loans. The time span of 8 years is inline with that in \citet{Uribe-Schmitt-Grohe-textbook}.}
Multiplying this by an average of 37\% haircut\footnote{The average sovereign haircut between 1970 and 2010 \citep{Cruces-Trebesch-13}.} implies that about 16.28\% of the debt is unsecured annually\footnotemark{}.
\footnotetext{In the mode, we assume that the country defaults on 100\% of the debt, hence this approach is necessary to handle the case of a haircut.}
Since we are dealing with a model with quarterly period, this results in the 65\% debt-to-GDP ratio targeted during calibration.
The average frequency of default is justified by the fact that Sri Lanka experienced its first default on 2022 since its independence in 1948; this give an average of 1.37 times per century\footnote{$\frac{1}{2022-1948} \times 100 \approx 1.37$}. Finally, (?)
% following the default of Sri Lanka since April 2022, the GDP growth is -7.4\%, -11.5\%, and -12.4\%, for the subsequent three quarters\footnote{Data Source: Central Bank of Sri Lanka}. This implies a geometric mean of -10.5\%.
Table \ref{tab:cal-sri-lanka} summarizes the calibrated parameters and their sources.



\subsection*{Calibration of Pakistan}
The calibration strategy for Pakistan is similar to that of Sri Lanka.
The parameters for the output process is obtained from the cyclical component of the HP-filter on the annual log-real-GDP for Pakistan from 1980 to 2021, which yields $\rho = $ 0.9008 and $\sigma_u=$ 0.0111 (see \autoref{fig:decompose-gdp}).
The risk-free interest rate remains to be 4\% annually, hence $r= $1\%.
Pakistan defaulted on January 1999, completed its debt restructuring on December 1999 \citep{SPGlobal-default-report}, but gained partial reaccess (flows > 0) in 2004, and full reaccess (flow > 1\% of GDP) in 2006 \citep*[][Table 5.6]{trebesch-2011-sovereign}.
The model adopted in my thesis does not distinguish between partial or full reaccess, hence the reentry period is set as the first year Pakistan gain positive flow of debt. Accordingly, the reentry period is set to 6 years (24 quarters), and $\theta=$ 0.0417.

The labor share is set as 0.4 to match the capital share in real GDP, following \citet{Pakistan-DSGE-calibration}. The share of tradable consumption is calibrated according to the tradable-to-GDP ratio over 1980 to 2021, which gives $a=$ 0.33. The intratemporal elasticity of substitution of consumption $\xi=$ 0.5 and $\sigma=$ 2, following the same justification for Sri Lanka \citep{Pakistan-DSGE-calibration,Uribe-Schmitt-Grohe-textbook}.
Nominal wage rigidity is set as $\gamma=$ 1.048 based on the empirical estimation of downward wage rigidity in 2014 by \citet*{wage-rigidity-data}.

Finally, the triplet $\left( \beta, \delta_1, \delta_2 \right)$ is also chosen to match the three equilibrium results:
\begin{enumerate*}[label = (\roman*)]
    \item the average debt-to-traded-GDP ratio in periods of good standing is 18\% per quarter;
    \item the frequency of default is 4 times per century; and
    \item the average output loss is 7\% per year conditional on being in financial autarky.
\end{enumerate*}
\begin{enumerate}[label = (\roman*)]
    \item
    The average debt-to-GDP-ratio between 2006 and 2013\footnote{
    Following the same logic as with Sri Lanka, I choose an 8-years window before the increasing amount of loans from China in 2013.}
    is 30\% according to the International Debt Statistics. Multiplying it with a 15\% haircut ratio estimated in \citet{Cruces-Trebesch-13} yields a 4.5\% of annual unsecured debt, which gives an 18\% unsecured debt-to-GDP ratio quarterly.
    \item History defaults or restructurings in Pakistan according to the BoC-BoE Sovereign Default Database states that Pakistan default (or rescheduled) to IMF 2 times (1978, 1980), to Paris Club 5 times (1972, 1974, 1981, 1999, 2001)\footnote{
        These defaults differ. In 1972, 1974 and 2001, Pakistan was dealt with under \emph{ad hoc terms}, meaning that the creditors is able to fix the terms and conditions for restructuring or rescheduling; in 1981, Pakistan was treated under the \emph{Classic terms}, in which it received a waiver on interest; in 1999, which is the default episode most widely recorded, Pakistan was dealt with under the \emph{Houston terms}, which offers an even longer grace period \citep{pakistan-default-start}.
    }, and to China 2 times (2002, 2021). This yields an average frequency of 4 times per century per creditor\footnote{$\frac{2+5+2}{2022 - 1947}\times 100 \times \frac{1}{3}$}.
    \item
    Following calculations based on \citet*{zarazaga-12}, the capital-output ratio dropped from 1.43 to 1.35 after the default episode. This yields an output cost of 5\% each year if the loss is attributed solely on the default in 1999. See Appendix \ref{ap: zarazaga} for more detail.
\end{enumerate}
Table \ref{tab:cal-pakistan} summarizes the calibrated parameters for Pakistan.


\section{Numerical Computation}
\label{sec:computation}
The approximated equilibrium is obtained by conducting value function iteration over an $n_y \times n_d$ discretized and equally spaced state space,
where $n_y = $ 200 is the number of grids for the output process and $n_d=$ 200 is the number of grids for the debt \citep{Na-18}. Denote $[\underline{y}^T, \overline{y}^T]$ as the lower and upper bound of output grid. Following \citet{Uribe-Schmitt-Grohe-textbook}, this is set as $[-4.2 \sigma_u, 4.2 \sigma_u]$. Also following the authors, since the average debt levels for both countries do not exceed 150\%, the upper bound for debt is set as 1.5, therefore the debt range for the value function iteration is $[\underline{d}, \overline{d}]=[0,1.5]$.

I follow \citet{Schmitt-Uribe-16} and discretize the AR(1) process for output by constructing a transition probability matrix over the grids of the output.
A time series of 10 million observations was generated based on \refeq{eq:ar1-output}. Each observation is then assigned to the nearest grid point among the 200 discrete values of $\ln y^T$. The discretized series is analyzed to calculate the probabilities of transitioning from one discrete state to another in consecutive periods.
To obtain the transition probability matrix, a $200\times200$ matrix is initialized with zeros. For each pair of consecutive observations, the corresponding element in the matrix is incremented by 1. After considering all the observations, the matrix is normalized by dividing each row by the sum of its elements. This results in the estimated transition probability matrix, which effectively captures the covariance matrices of order 0 and 1
\citep*{Uribe-Schmitt-Grohe-textbook}.

Value function iteration is then conducted for a given set of calibrated parameters, which is composed of the set of fixed parameters $(\rho, \sigma_u, r^*, \theta, \alpha, a, \xi, \gamma)$ and the tuple $(\beta, \delta_1, \delta_2)$. The default dynamic and moments can then be conducted once the value function converges. Following \citet{Uribe-Schmitt-Grohe-textbook} and \citet{Na-18}, a simulation of the model based on the policy function is conducted sequentially\footnote{
    The main randomness comes from the AR(1) output. Once $y^T_{t+1}$ is determined, along with determination of the debt level for next period $d_{t+1}$, we have the state $(y^T_{t+1}, d_{t+1})$ for $t+1$, which is a grid-point. Values for the rest of the endogenous variables such as $c_{t+1}, I_{t+1}$ can then be obtained by extracting the corresponding value from the state grid-point on the policy function.
} for 1.1 million iterations, of which the first 0.1 million periods are discarded.

A critical procedure of the calibration is to select the tuple $(\beta, \delta_1, \delta_2)$ such that it matches the three targets described in section \ref{sec: calibration}, which are the default frequency, debt-to-GDP ratio, and output loss.
In particular, the tuple is chosen to minimize the Euclidean distance between the target value and the corresponding moments generated by the dynamic simulation\footnote{
    Default frequency is evaluated by taking the mean of $\left\{ I_{t} \right\}_{t \in T}$ and multiply it by $100\times 4$ quarters per century. Debt-to-GDP ratio is calculated by taking the mean of $\left\{ d_t / y^T_t \right\}_{t \in \text{Good T}}$, which is the average debt-to-tradable-output ratio condition on being in good standing. Output loss is evaluated by taking the mean of $\left\{ (y^T_t - \tilde{y}^T_t)/y^T_t \right\}_{t \in \text{Bad T}}$, which is the average output difference between the endowment and the actual received output, divided by the endowment, over periods of bad standings.
}. Since the value function iteration and dynamic simulation are time-consuming processes, I use the surrogate optimization algorithm in MATLAB as the searching algorithm, as it is suitable for minimizing time-consuming objective functions.

The estimated results are reported in Table~\ref{tab: calibration-compare}.
The upper part of the table presents the estimated values of $(\beta, \delta_1, \delta_2)$, and the lower part of the table shows the difference between the target and simulated moments under the optimal calibration (see the rows where the first column is ``HP'').
There exist errors in the minimal distance which is inevitable upon the computation. For Sri Lanka, the debt-to-tradable-GDP ratio is 1.73, lower than the target by 0.02, and the output loss is higher by 3\%. Frequency of default per century gives 1.26, which is about half the frequency I target. Similar results are obtained in the case of Pakistan. The debt-to-tradable-GDP ratio fits pretty well, and the output loss is lower by 1.3\%, but the frequency of default is also 1.26 times per century, also about have of the targeted default frequency.\footnote{
    With that being said, the results are actually pretty satisfying. The target of defaulting 2.6 times per century for both countries is set due to the ambiguity of determining default episodes, so I target a benchmark value of 2.6 times. However, Sri Lanka officially declared default for the first time in 2022, which yields $1/(2023-1948) \times 100 \approx 1.3$ times per century. As for Pakistan, \citet{Uribe-Schmitt-Grohe-textbook} and \citet{SPGlobal-default-report} only record the default event in 1999, which up to now yields $1/(2023-1947) \times 100 \approx 1.3$ times per century. These calculations are naive and inexact, but are roughly able to justify the estimated value of frequency of 1.26. A more robust target requires longer historical data throughout the history, which I am unable to tackle in this thesis.
}

\section{Results}