Typically, a model under the Eaton-Gersovitz framework does not have an analytical solution. Therefore, the optimal default set defined by \refeq{eq:default-set}, as well as the value functions and the policy functions, must be obtained numerically via the technique of value function iteration.
This requires the assignment of functional forms as well as structural parameters that matches the economy.
I follow the functional forms and the calibration approach introduced in \citet{Na-18} and \citet{Hinrichsen_2020-chapter4}.

\section{Calibration}
\label{sec: calibration}
\subsection*{Functional Forms}
Following \citet{Na-18}, the time unit is assumed to be one quarter, and the periodic utility function is assumed to be the constant relative risk aversion (CRRA) type
\begin{equation}
    \label{eq:CRRA-utility}
    U(c_t) = \frac{c_t^{1-\sigma} - 1}{1 - \sigma},
\end{equation}
where $\sigma$ is the inverse of elasticity of intertemporal substitution of the consumption.
The aggregator function for tradable and non-tradable consumption takes the constant elasticity of substitution (CES) form
\begin{equation}
    \label{eq:aggregator-function}
    c_t = A(c^T_t, c^N_t) =
        \left[ a \left( c^T_t \right)^{1- \frac{1}{\xi}} +
            (1 - a) \left( c^N_t \right)^{1- \frac{1}{\xi}}
        \right]^{\frac{1}{1 - \frac{1}{\xi}}}.
\end{equation}
The CES aggregator states that the share of tradable consumption is $a \in [0,1]$, and the elasticity of substitution between the tradable and non-tradable consumption is $\xi$.
Moreover, following the literature, to make the consumption of tradable goods $c^T_t$ and the external debt $d_t$ independent of the outputs in the nontradable sector in the equilibrium,
assume that the inter- and intratemporal elasticity of substitution is equivalent \citep*[See][Chapter 9.5]{Uribe-Schmitt-Grohe-textbook}.
That is,
\begin{equation}
    \label{eq:xi-sigma}
    \xi = \frac{1}{\sigma}.
\end{equation}
The production technology for the nontradable goods follows a simple form
\begin{equation}
    \label{eq:production-function}
    y^N_t = F(h_t) = h_t ^\alpha.
\end{equation}
The loss-function in \refeq{eq:ytt} is positive and increasing with $y^T_t$, and following \citet{Chatterjee-12}, I adopt the quadratic form with two parameters
\begin{equation}
    L(y^T_t) = \max \left\{
        0, \delta_1 y^T_t + \delta_2 \left( y^T_t \right)^2
     \right\}.
\end{equation}
This is also adopted in \citet{Na-18}. In this setting, if we set $\delta_1 < 0$ and $\delta_2 >0$, the output-loss increases as $y^T_t$ increases, indicating that the more a country is endowed, the more it loses during default.

\subsection*{Calibration of Sri Lanka}
\label{sec: cal-sri}
The model is calibrated to Sri Lanka from 2007 to 2017, when the Chinese government started to provide the increasing amount of loans (see \autoref{fig: sri-lanka-debt-ts}).
The output process of \refeq{eq:ar1-output} is proxied by the measure of tradable GDP. I follow \citet{Schmitt-Uribe-16} in obtaining tradable outputs as the sum of GDP in agriculture, forestry, fishing, mining and manufacturing. Considering that the seasonality in the quarterly data for Sri Lanka might impose a higher volatility estimated in the AR(1) process, I follow \citet{Hinrichsen_2020-chapter4} and estimate the annual data over 1980 -- 2021. I obtain the cyclical component of the tradable outputs by filtering this time series with an HP-filter with the smoothing parameter set to $100$.
Estimation of the AR(1) on the cyclical component thus yields $\rho = (0.93)$ and $\sigma_u = (0.037)$.

The global risk-free world interest rate $r^*$ is set to match the 3-month treasury bill rate during the period. The 3-month treasury bill experienced a continuous low rate since 2008, with a median of $0.12\%$ annually. This gives a quarterly risk-free rate of $0.03\%$.
The probability of reentry is difficult to assess since Sri Lanka encountered its first default in April 12, 2022, and is not yet undergoing the process of restructuring. As a result, following \citet*{Chatterjee-12} and \citet*{Hinrichsen_2020-chapter4}, I set the probability of reentry to $0.0385$, which implies that the country will be in default on average for about $6.5$ years, matching the median of default spans for past 100 systemic crises \citep*{Reinhart-Rogoff-2014-100-episode}.

Calibration on other structural parameter regarding the Sri Lanka economy follows \citet*{Jegajeevan-Sri-Lanka-DSGE}. The author estimates the Sri Lanka economy with a DSGE model. Labor share

\subsection*{Calibration of Pakistan}
\label{sec: cal-pak}
The calibration strategy for Pakistan is similar to that of Sri Lanka.
The parameters for the output process is obtained from the cyclical component of the HP-filter on the annual log-real-GDP for Pakistan from 1980 to 2021, which yields $\rho = $ 0.9008 and $\sigma_u=$ 0.0111 (see \autoref{fig:decompose-gdp}).
The risk-free interest rate remains to be 4\% annually, hence $r= $1\%.
Pakistan defaulted on January 1999, completed its debt restructuring on December 1999 \citep{SPGlobal-default-report}, but gained partial reaccess (flows > 0) in 2004, and full reaccess (flow > 1\% of GDP) in 2006 \citep*[][Table 5.6]{trebesch-2011-sovereign}.
The model adopted in my thesis does not distinguish between partial or full reaccess, hence the reentry period is set as the first year Pakistan gain positive flow of debt. Accordingly, the reentry period is set to 6 years (24 quarters), and $\theta=$ 0.0417.

The labor share is set as 0.4 to match the capital share in real GDP, following \citet{Pakistan-DSGE-calibration}. The share of tradable consumption is calibrated according to the tradable-to-GDP ratio over 1980 to 2021, which gives $a=$ 0.33. The intratemporal elasticity of substitution of consumption $\xi=$ 0.5 and $\sigma=$ 2, following the same justification for Sri Lanka \citep{Pakistan-DSGE-calibration,Uribe-Schmitt-Grohe-textbook}.
Nominal wage rigidity is set as $\gamma=$ 1.048 based on the empirical estimation of downward wage rigidity in 2014 by \citet*{wage-rigidity-data}.

Finally, the triplet $\left( \beta, \delta_1, \delta_2 \right)$ is also chosen to match the three equilibrium results:
\begin{enumerate*}[label = (\roman*)]
    \item the average debt-to-traded-GDP ratio in periods of good standing is 18\% per quarter;
    \item the frequency of default is 4 times per century; and
    \item the average output loss is 7\% per year conditional on being in financial autarky.
\end{enumerate*}
\begin{enumerate}[label = (\roman*)]
    \item
    The average debt-to-GDP-ratio between 2006 and 2013\footnote{
    Following the same logic as with Sri Lanka, I choose an 8-years window before the increasing amount of loans from China in 2013.}
    is 30\% according to the International Debt Statistics. Multiplying it with a 15\% haircut ratio estimated in \citet{Cruces-Trebesch-13} yields a 4.5\% of annual unsecured debt, which gives an 18\% unsecured debt-to-GDP ratio quarterly.
    \item History defaults or restructurings in Pakistan according to the BoC-BoE Sovereign Default Database states that Pakistan default (or rescheduled) to IMF 2 times (1978, 1980), to Paris Club 5 times (1972, 1974, 1981, 1999, 2001)\footnote{
        These defaults differ. In 1972, 1974 and 2001, Pakistan was dealt with under \emph{ad hoc terms}, meaning that the creditors is able to fix the terms and conditions for restructuring or rescheduling; in 1981, Pakistan was treated under the \emph{Classic terms}, in which it received a waiver on interest; in 1999, which is the default episode most widely recorded, Pakistan was dealt with under the \emph{Houston terms}, which offers an even longer grace period \citep{pakistan-default-start}.
    }, and to China 2 times (2002, 2021). This yields an average frequency of 4 times per century per creditor\footnote{$\frac{2+5+2}{2022 - 1947}\times 100 \times \frac{1}{3}$}.
    \item
    Following calculations based on \citet*{zarazaga-12}, the capital-output ratio dropped from 1.43 to 1.35 after the default episode. This yields an output cost of 5\% each year if the loss is attributed solely on the default in 1999. See Appendix \ref{ap: zarazaga} for more detail.
\end{enumerate}
Table \ref{tab:cal-pakistan} summarizes the calibrated parameters for Pakistan.


\section{Numerical Computation}
\label{sec:computation}
The approximated equilibrium is obtained by conducting value function iteration(VFI) over an $n_y \times n_d$ discretized and equally spaced state space, where $n_y = $ 200 is the number of grids for output process and $n_d=$ 200 is the number of grid for debt \citep{Na-18}. Denote $[\underline{y}^T, \overline{y}^T]$ as the lower and upper bound of output grid. Following \citet{Uribe-Schmitt-Grohe-textbook}, this is set as $[-4.2 \sigma_u, 4.2 \sigma_u]$. Also following the authors, since the average debt levels for both countries do not exceed 150\%, the upper bound for debt is set as 1.5, therefore the debt range for the VFI is $[\underline{d}, \overline{d}]=[0,1.5]$.

Due to the continuous assumption of the AR(1) process of $y^T_t$ (since we assume $\mu_t$ to be normal), the discretized method used in VFI is not directly applicable. \citet{Schmitt-Uribe-16} and \citet{Na-18} deal with this issue by constructing a transition probability matrix over the grids of the AR(1) output process.
A time series of 10 million observations was generated based on \refeq{eq:ar1-output}. Each observation was then assigned to the nearest grid point among the 200 discrete values of $\ln y^T$. The discretized series was analyzed to calculate the probabilities of transitioning from one discrete state to another in consecutive periods.
To obtain the transition probability matrix, a $200\times200$ matrix was initialized with zeros. For each pair of consecutive observations, the corresponding element in the matrix was incremented by 1. After considering all the observations, the matrix was normalized by dividing each row by the sum of its elements. This resulted in the estimated transition probability matrix, which effectively captured the covariance matrices of order 0 and 1.
\citep*{Uribe-Schmitt-Grohe-textbook}.


%%Value function Interation?
The equilibrium dynamics can be simulated once the VFI is conducted. Following \citet{Schmitt-Uribe-16}
and \citet{Na-18}, a simulation of based on the policy function is conduct 1.1 million time. After discarding the first 0.1 million periods, the periods in which a default occurs are identified, and a window of 12 quarters prior to and 12 quarters after each default episode is extracted. The median is then computed period by period across all windows, and the period of default is normalized to 0.

\section{Results}