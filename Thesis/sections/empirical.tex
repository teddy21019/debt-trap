Typically, a model under the Eaton-Gersovitz framework does not have an analytical solution. Therefore, the optimal default set defined by \refeq{eq:default-set}, as well as the value functions and the policy functions, must be obtained numerically via the technique of value function iteration.
This requires the assignment of functional forms as well as structural parameters that matches the economy.
I follow the functional forms and the calibration approach introduced in \citet{Na-18} and \citet{Hinrichsen_2020-chapter4}.

\section{Calibration}
\label{sec: calibration}

\subsection{Functional Forms}
Following \citet{Na-18}, the time unit is assumed to be one quarter, and the periodic utility function is assumed to be the constant relative risk aversion (CRRA) type
\begin{equation}
    \label{eq:CRRA-utility}
    U(c_t) = \frac{c_t^{1-\sigma} - 1}{1 - \sigma},
\end{equation}
where $\sigma$ is the inverse of elasticity of intertemporal substitution of the consumption.
The aggregator function for tradable and non-tradable consumption takes the constant elasticity of substitution (CES) form
\begin{equation}
    \label{eq:aggregator-function}
    c_t = A(c^T_t, c^N_t) =
        \left[ a \left( c^T_t \right)^{1- \frac{1}{\xi}} +
            (1 - a) \left( c^N_t \right)^{1- \frac{1}{\xi}}
        \right]^{\frac{1}{1 - \frac{1}{\xi}}}.
\end{equation}
The CES aggregator states that the share of tradable consumption is $a \in [0,1]$, and the elasticity of substitution between the tradable and non-tradable consumption is $\xi$.
Moreover, following the literature, to make the consumption of tradable goods $c^T_t$ and the external debt $d_t$ independent of the outputs in the nontradable sector in the equilibrium,
assume that the inter- and intratemporal elasticity of substitution is equivalent \citep*[See][Chapter 9.5]{Uribe-Schmitt-Grohe-textbook}.
That is,
\begin{equation}
    \label{eq:xi-sigma}
    \xi = \frac{1}{\sigma}.
\end{equation}
The production technology for the nontradable goods follows a simple form
\begin{equation}
    \label{eq:production-function}
    y^N_t = F(h_t) = h_t ^\alpha.
\end{equation}
The loss-function in \refeq{eq:ytt} is positive and increasing with $y^T_t$, and following \citet{Chatterjee-12}, I adopt the quadratic form with two parameters
\begin{equation}
    L(y^T_t) = \max \left\{
        0, \delta_1 y^T_t + \delta_2 \left( y^T_t \right)^2
     \right\}.
\end{equation}
This is also adopted in \citet{Na-18}. In this setting, if we set $\delta_1 < 0$ and $\delta_2 >0$, the output-loss increases as $y^T_t$ increases, indicating that the more a country is endowed, the more it loses during default.

\subsection{Calibration of Sri Lanka}
The model is calibrated to Sri Lanka from 2007 to 2017, when the Chinese government started to provide the increasing amount of loans (see \autoref{fig: sri-lanka-debt-ts}).
The output process of \refeq{eq:ar1-output} is proxied by the measure of tradable GDP. I follow \citet{Schmitt-Uribe-16} in obtaining tradable outputs as the sum of GDP in agriculture, forestry, fishing, mining and manufacturing. Considering that the seasonality in the quarterly data for Sri Lanka might impose a higher volatility estimated in the AR(1) process, I follow \citet{Hinrichsen_2020-chapter4} and estimate the annual data over 1980 -- 2021. I obtain the cyclical component of the tradable outputs by filtering this time series with an HP-filter with the smoothing parameter set to $100$.
Estimation of the AR(1) on the cyclical component thus yields $\rho = (0.93)$ and $\sigma_u = (0.037)$.

The global risk-free world interest rate $r^*$ is set to match the 3-month treasury bill rate during the period. The 3-month treasury bill experienced a continuous low rate since 2008, with a median of $0.12\%$ annually. This gives a quarterly risk-free rate of $0.03\%$.
The probability of reentry is difficult to assess since Sri Lanka encountered its first default in April 12, 2022, and is not yet undergoing the process of restructuring. As a result, following \citet*{Chatterjee-12} and \citet*{Hinrichsen_2020-chapter4}, I set the probability of reentry to $0.0385$, which implies that the country will be in default on average for about $6.5$ years, matching the median of default spans for past 100 systemic crises \citep*{Reinhart-Rogoff-2014-100-episode}.

Calibration on other structural parameter regarding the Sri Lanka economy follows \citet*{Jegajeevan-Sri-Lanka-DSGE}. The author estimates the Sri Lanka economy with a DSGE model. Labor share

\subsection{Calibration of Pakistan}
The calibration strategy for Pakistan is similar to that of Sri Lanka.
The parameters for the output process is obtained from the cyclical component of the HP-filter on the annual log-real-GDP for Pakistan from 1980 to 2021, which yields $\rho = $ 0.9008 and $\sigma_u=$ 0.0111 (see \autoref{fig:decompose-gdp}).
The risk-free interest rate remains to be 4\% annually, hence $r= $1\%.
Pakistan defaulted on January 1999, completed its debt restructuring on December 1999 \citep{SPGlobal-default-report}, but gained partial reaccess (flows > 0) in 2004, and full reaccess (flow > 1\% of GDP) in 2006 \citep*[][Table 5.6]{trebesch-2011-sovereign}.
The model adopted in my thesis does not distinguish between partial or full reaccess, hence the reentry period is set as the first year Pakistan gain positive flow of debt. Accordingly, the reentry period is set to 6 years (24 quarters), and $\theta=$ 0.0417.

The labor share is set as 0.4 to match the capital share in real GDP, following \citet{Pakistan-DSGE-calibration}. The share of tradable consumption is calibrated according to the tradable-to-GDP ratio over 1980 to 2021, which gives $a=$ 0.33. The intratemporal elasticity of substitution of consumption $\xi=$ 0.5 and $\sigma=$ 2, following the same justification for Sri Lanka \citep{Pakistan-DSGE-calibration,Uribe-Schmitt-Grohe-textbook}.
Nominal wage rigidity is set as $\gamma=$ 1.048 based on the empirical estimation of downward wage rigidity in 2014 by \citet*{wage-rigidity-data}.

Finally, the triplet $\left( \beta, \delta_1, \delta_2 \right)$ is also chosen to match the three equilibrium results:
\begin{enumerate*}[label = (\roman*)]
    \item the average debt-to-traded-GDP ratio in periods of good standing is 18\% per quarter;
    \item the frequency of default is 4 times per century; and
    \item the average output loss is 7\% per year conditional on being in financial autarky.
\end{enumerate*}
\begin{enumerate}[label = (\roman*)]
    \item
    The average debt-to-GDP-ratio between 2006 and 2013\footnote{
    Following the same logic as with Sri Lanka, I choose an 8-years window before the increasing amount of loans from China in 2013.}
    is 30\% according to the International Debt Statistics. Multiplying it with a 15\% haircut ratio estimated in \citet{Cruces-Trebesch-13} yields a 4.5\% of annual unsecured debt, which gives an 18\% unsecured debt-to-GDP ratio quarterly.
    \item History defaults or restructurings in Pakistan according to the BoC-BoE Sovereign Default Database states that Pakistan default (or rescheduled) to IMF 2 times (1978, 1980), to Paris Club 5 times (1972, 1974, 1981, 1999, 2001)\footnote{
        These defaults differ. In 1972, 1974 and 2001, Pakistan was dealt with under \emph{ad hoc terms}, meaning that the creditors is able to fix the terms and conditions for restructuring or rescheduling; in 1981, Pakistan was treated under the \emph{Classic terms}, in which it received a waiver on interest; in 1999, which is the default episode most widely recorded, Pakistan was dealt with under the \emph{Houston terms}, which offers an even longer grace period \citep{pakistan-default-start}.
    }, and to China 2 times (2002, 2021). This yields an average frequency of 4 times per century per creditor\footnote{$\frac{2+5+2}{2022 - 1947}\times 100 \times \frac{1}{3}$}.
    \item
    Following calculations based on \citet*{zarazaga-12}, the capital-output ratio dropped from 1.43 to 1.35 after the default episode. This yields an output cost of 5\% each year if the loss is attributed solely on the default in 1999. See Appendix \ref{ap: zarazaga} for more detail.
\end{enumerate}
Table \ref{tab:cal-pakistan} summarizes the calibrated parameters for Pakistan.


\section{Numerical Computation}
\label{sec:computation}
The approximated equilibrium is obtained by conducting value function iteration(VFI) over an $n_y \times n_d$ discretized and equally spaced state space, where $n_y = $ 200 is the number of grids for output process and $n_d=$ 200 is the number of grid for debt \citep{Na-18}. Denote $[\underline{y}^T, \overline{y}^T]$ as the lower and upper bound of output grid. Following \citet{Uribe-Schmitt-Grohe-textbook}, this is set as $[-4.2 \sigma_u, 4.2 \sigma_u]$. Also following the authors, since the average debt levels for both countries do not exceed 150\%, the upper bound for debt is set as 1.5, therefore the debt range for the VFI is $[\underline{d}, \overline{d}]=[0,1.5]$.

Due to the continuous assumption of the AR(1) process of $y^T_t$ (since we assume $\mu_t$ to be normal), the discretized method used in VFI is not directly applicable. \citet{Schmitt-Uribe-16} and \citet{Na-18} deal with this issue by constructing a transition probability matrix over the grids of the AR(1) output process.
A time series of 10 million observations was generated based on \refeq{eq:ar1-output}. Each observation was then assigned to the nearest grid point among the 200 discrete values of $\ln y^T$. The discretized series was analyzed to calculate the probabilities of transitioning from one discrete state to another in consecutive periods.
To obtain the transition probability matrix, a $200\times200$ matrix was initialized with zeros. For each pair of consecutive observations, the corresponding element in the matrix was incremented by 1. After considering all the observations, the matrix was normalized by dividing each row by the sum of its elements. This resulted in the estimated transition probability matrix, which effectively captured the covariance matrices of order 0 and 1.
\citep*{Uribe-Schmitt-Grohe-textbook}.


%%Value function Interation?
The equilibrium dynamics can be simulated once the VFI is conducted. Following \citet{Schmitt-Uribe-16}
and \citet{Na-18}, a simulation of based on the policy function is conduct 1.1 million time. After discarding the first 0.1 million periods, the periods in which a default occurs are identified, and a window of 12 quarters prior to and 12 quarters after each default episode is extracted. The median is then computed period by period across all windows, and the period of default is normalized to 0.

\section{Default Set}
\label{sec: default-set}
As described in Section \ref{sec: model-default-decision} and Appendix \ref{ap: default-set}, the default decision is jointly decided by the current debt level and its output for tradable goods during the period. Under a certain debt stock $d_t$, the less the tradable output $y^T_t$ it encounters, the lower the benefit a country receives.
Ultimately, if the output falls below a certain threshold value, it becomes optimal for the country to default.
\citet{Hinrichsen_2020-chapter4} compares the theoretical scenarios of defaulting with the actual data with a graphical default. He maps the default set evaluated via value function iteration into the state space, with the output as the horizontal axis and the debt level as the vertical axis. A grid on the space is colored in gray if the corresponding debt-output pair is \emph{not} in the default set. Figure~\ref{fig:ds} demonstrates the default set for Sri Lanka and Pakistan. The debt level in the default set plot differs between the two countries due to the difference in the targeted equilibrium debt-to-tradable-GDP ratio. As shown in the figure and proven in Appendix \ref{ap: default-set}, the upper limit of a default set (the intersection of the white and gray areas given a certain debt level) increases as the debt level increases, indicating that it requires a higher endowment of tradable output in order to incentivize the country to continue to fulfill its obligations to repay\footnote{
    In the default set depicted in \citet{Hinrichsen_2020-chapter4}, there is a discontinuity in the upper bound of the default set. This indicates that when the output level drops to a certain threshold, the country chooses to default, regardless of the current debt level. This is contradicted to the analytical conclusion proved in Proposition \ref{prop3} in Appendix \ref{ap: default-set}. During a private correspondence, the author acknowledged this mistake in his doctoral thesis, but he has already corrected it in his latest book.
}.

The actual data is then plotted on the default set for comparison. The output level corresponds to the cyclical component of the per capita tradable-GDP obtained by the HP-filter for the respective year.
The corresponding debt level is calculated by multiplying the debt-to-nominal-tradable-GDP ratio by the haircut ratio and 4 quarters. This calculation is necessary because the targeted debt level during calibration specifically refers to the unsecured portion of the debt, and thus the data should undergo the same calculation to accurately represent the unsecured debt component.

Furthermore, the debt data used in this analysis consists of two components: the debt stock excluding China obtained from the International Debt Statistics (IDS), and the debt stock specifically for China sourced from the database created by \citet*{Horn-Reinhart-Trebesch-21}. It is worth noting that Chinese government loans channeled through state-owned enterprises are typically not reported in the World Bank Debt Reporting System (DRS), which is the basis for IDS. This omission leads to an underestimation of the actual debt burden faced by the country. The additional loans provided by state-owned commercial banks, referred to as ``hidden debts'' by \citet*{Horn-Reinhart-Trebesch-21}, are included in their database. Consequently, in order to capture the total debt burden accurately, the total debt stock reported in IDS is adjusted by excluding the debt to China and adding the debt to China obtained from the database by \citet*{Horn-Reinhart-Trebesch-21}.

\subsection*{Default Decision for Sri Lanka}

Figure \ref{fig: ds-sri-data-with-china} plots the debt-output pair for Sri Lanka from 2007 to 2017. Each point represents the actual data pair for the year. Recall that China initiated its constructions in infrastructures such as the Hambantota International Port in 2007, and Mattala Rajapaksa International Airport in 2009. The selection of this time span allows us to examine the change in debt burden for Sri Lanka comprehensively in a chronological way.

As shown in the figure, at the initial phase of China's investment in 2007 and 2008, Sri Lanka is still under its safe zone of not defaulting, despite that it experienced a little drop in the output. Sri Lanka first landed into the default set during 2009, as debt to China reached \$3 billion. During 2010 and 2011, Sri Lanka ended its 25-years civil war, and experienced an unprecedented output growth. Agriculture sector in 2010 grew for 7\%, and industry sector grew for 8\%. This rapid increasing in GDP largely lowered the debt-to-tradable-GDP ratio, both in 2010 and 2011, and hence Sri Lanka was able to avoid the risk of default and stayed under the non-default set. However, as GDP growth in tradable good slowed down to a rate of 5\%, the debt to China increased by 6\%, which slowly pushed Sri Lanka to the brink of default. Eventually, in 2013, as total debt reached \$41 billion, in which China accounted for 13\%, Sri Lanka was again entering the default set. The status continued until 2016, with 2014 being an ambiguous exception.

It is worth mentioning that in 2015, Maithripala Sirisena defeated Mahinda Rajapaksa in presidential election, and promised to extricate Sri Lanka from the debt burden from China. However, according to Figure \ref{fig: ds-sri-data-with-china}, Sri Lanka was too indebted during 2015 and 2016, and it would be difficult for the new government to take actions besides undergoing debt restructuring. Eventually, in 2017, the infamous event of Sri Lanka leasing the Hambantota International Port for 99 years and selling 70\% of the stake to China Merchants Port stroke the headline. This action, however, did not write off the loans for the construction of the infrastructures, as it is used as a raise in money to repay debt to other official creditors \citep{Brautigam-meme-2020, Moramudali_2019}. Regardless of the narrative of whether the lease is involuntary or not, along with the outstanding output level in 2017, Sri Lanka was back to the non-default set after the lease.
Observe that all the cyclical components of the tradable goods (the x coordinate) are above 1 after 2011. This indicates that Sri Lanka is under the default set even when the output is above the steady state level of tradable outputs.

An intriguing question arises: What if the debt to China is not included, holding other conditions fixed? \citet{Hinrichsen_2020-chapter4} compares the debt for countries in war reparations with and without indemnities, and here I adopt the same approach. Figure \ref{fig: ds-sri-data-x-china} shows the comparison of the data. In the figure, the solid black dots represent data points including debts owed to China, while the gray dots represent data points excluding debts from China.
It is obvious that, \emph{ceteris paribus}, all years spanning from 2009 to 2017 are not under the default set. This is, indeed, an inexact comparison.
First, the debt level given in the model is endogenous, and we do not distinguish between China and other lenders. It is possible that Sri Lanka sought for other creditors for loans if the infrastructures were planned voluntarily, as mentioned in \citet{Brautigam-meme-2020}. In this case the debt level in the absence of China can be underestimated.
Second, the impact on GDP would be uncertain if China had not been involved in the construction of the infrastructure projects. For instance, \citet{Bandiera-Vasileios-BRI-debt} estimates from a long-term growth model that in 2020, the additional growth from investment on the Belt and Road Initiative (BRI) is about 0.08\%.
Estimations of the extra growth due to investments during 2009 to 2017 are crucial in order to examine the counterfactual debt burden in the absence of China's intervention. If investments in China indeed contributed to the GDP growth during the periods, the debt-to-tradable-GDP ratio is also underestimated.

\subsubsection*{Default Decision for Pakistan}



\section{Robustness Check}

\subsection{Default Probability}
In the model of \citet{Na-18}, the price of debt offered by foreign lenders in period $t$ takes the probability of the economy defaulting in the next period into consideration, given that the economy does not default in the current period and is in good financial standing, as depicted in \refeq{eq:lender}.
As a result, it is worthy to also examine the conditional probability of defaulting under the current state.\footnote{%
    As suggested by several committees, the ``unconditional'' default probabilities under states are able to provide more insights to the issue of debt trap compared to a clear-cut region of the default set. However, the model is unable to evaluate such concept as the default decision is deterministic under given state variables. In turn, I report the probability of defaulting in the consecutive period ``conditional'' on the economy is currently in good financial standing. This provides a slightly different interpretation on the risk of defaulting, but is still able to grasp the concept of uncertainty under different states of an economy.
}

The conditional probability of default $\Pr (I_{t+1} = 0 \mid I_{t} = 1)$ is equivalent to the probability that the output in the next period does not fall into the default set:
\begin{equation*}
    \Pr (y_{t+1} \in D(d_{t+1}) \mid y^T_t)
\end{equation*}
In numerical computation, this is obtained by multiplying the transition probability matrix of output with the default policy function, which is a matrix over all state space of output and debt.

Figure \ref{fig: dp} demonstrates the default probability over each state in the model for both Sri Lanka and Pakistan. The darker region represents a lower probability of defaulting in the next period, condition on the debt level unchanged. Notice that for some states that are under the default set in Figure \ref{fig:ds}, the default probability might not be 100\%. This is because if for some reason the economy is unable to default (even if it is optimal to do so), there is a certain probability that the output shocks upward, and the economy moves back to the non-default set. As a result, Figure \ref{fig: dp} provides a fuzzy set of default risk among the boundary of default set.
The output-debt ratios form the observed data as well as the datapoints that remove the debt to China are also shown in Figure \ref{fig: dp}. The calculations are equivalent to those in Section \ref{sec: default-set}.

Reexamining the datapoints of Sri Lanka under Figure \ref{fig: dp-sri}, it is obvious that during 2012 to 2017, Sri Lanka is wandering around the fuzzy region. Among all years that are in the default set, only 2015 and 2016 is under the level of high default risk. This justifies the conclusion that the new president of Sri Lanka is under great pressure on having to renegotiate with China or other creditors. When removing the debts to China, all years appear in the lowest level on the scale, indicating that the default probabilities in the absence of China are zero.

One might be curious how the default probability behave when debt from other creditor is removed in the case of Sri Lanka. From Figure \ref{fig: sri-lanka-debt-ts}, debt to China exceeds all other main creditors after 2013. A quick observation on Figure \ref{fig: dp-sri} suggested that excluding other creditors instead will in general lead to a higher debt-to-tradable-GDP ratio, and therefore the default probability will remain in the fuzzy zone of default.\footnote{%
    Take 2016 for instance. Removing the debt from the second-highest creditor(s), in this case the aggregated debt stock from members of the Paris Club, yields a debt-to-tradable-GDP ratio of 199\% (haircut and quarter adjusted), which corresponds to about 45\% chance of defaulting in the next period. Meanwhile, removing China yields a debt-to-tradable-GDP ratio of 187\%, which corresponds to merely 5\% chance of defaulting in the next period. Since the direction of ratio by removing other creditors is trivial and obvious, I do not list all the results on the graph.
}

As for the case of Pakistan, the results do not provide further insight compared that from the default set. In 2010 and 2017, the default probability is solidly 100\%, in-line with the conclusion elaborated in the previous section. Debt-ouptut pairs in the absence of China after 2015 yield almost 0\% probabilities of defaulting for all periods.\footnote{%
    Removing debt from the second-highest creditor, which is the World Bank, yields a debt-to-tradable-GDP ratio of 101\%. According to the figure, the probability is still close to zero.
}



\subsection{Filtering Method}
\input{sections/result-sec/filter_method.tex}