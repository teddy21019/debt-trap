\begin{acknowledgementszh}

這個論文能夠順利的完成 ,我首先要感謝我的指導老師,何泰寬教授。

這其實是我的第三個題目。第一個題目是透過agent-based model 模擬CBDC,但考慮時間後及時停損。第二個題目是關於薩爾瓦多採用比特幣作為法定貨幣對直接投資的影響,但因為中間的影響管道實在缺乏一個令人信服的經濟管道,所以在何老師的建議下更換為現在這個題目。此時已經五月初,因此若不是何老師在接下來三個月細心的協助跟指導,我無法想像我能在短時間之內完成我的論文。
老師對於科學方法的嚴謹,是我完成這篇論文後最大的收穫。剛接收到這個題目的第三周,我就急著糊裡糊塗的進行模型校準,跑了文獻中的程式碼,最後出來的結果慘不仁賭。
在一次的meeting之後當頭棒喝,發現我的理解有很大的錯誤。於是我重新細讀各種文獻,重新理解做法,仔細認識我的研究對象的背景,經過了將近一個月的理解與嘗試後,才終於如願地跑出這個令人振奮的結果。因為何老師對學術研究高規格的要求,我對社會科學的實證研究態度有了重大的轉變,我由衷的感激,也十分慶幸能接受何老師的指導。

感謝口試委員朱玉琦老師、葉國俊老師、楊宗翰老師、以及楊茜文老師對於論文內容的修改建議。各位老師對於本研究的想法與建議,都使這份研究更加具有學術價值。另外,雖然 agent-based model 最後沒有用於我的論文,我還是感謝陳樹衡老師開放旁聽有關此一工具的課程。

在幾次更換題目的期間,我的心情其實是十分低落且緊張的。我很感謝姵璇在這段時間對我的鼓勵以及陪伴,在這段期間願意聆聽我的各種苦訴,在我覺得好像什麼都做不好的時候願意跟我研究如何將實習時很冗的工作自動化,給我很大的信心。我也要感謝宗弘、霖東、以及 651室的張翔、呂越,隨時跟我分享一些新奇的經濟學想法;感謝鑫君做為好鄰居協助統計、偉駿與我共同研究數據處理還有程式設計、宇杰協助Latex的設定、以及Oscar 協助我進行西班牙語文件的翻譯。口試前一周,也感謝馥暄以及芳語協助練習。感謝系辦禮禎協助所有遠距口室相關事宜。我也要感謝我的家人,無論是什麼時候都在支持著我的決定,放心的讓我挑戰我先前從未接觸過的經濟。

在我的研究生涯中還有太多可以感謝的人,無論事情是大事小,我都很感激你們的存在。

I would like to express my gratitude to Dr. Hinrichsen for sharing the code and engaging in discussions regarding the default set graph. Without your generosity, I would not have been able to complete my thesis. Additionally, I extend my thanks to Dr. Na and Dr. Schmitt-Grohé for their insightful discussions on the model, which helped clarify important concepts for me. Lastly, I am truly thankful for Mr. Sperr's prompt reply and assistance in proofreading my thesis just a week before my oral defense.
\\[2em]
\begin{flushright}
    陳家威 僅誌\\
    於國立臺灣大學經濟學研究所

    中華民國一一二年七月二十日
\end{flushright}

\end{acknowledgementszh}

% \begin{acknowledgementsen}
% I'm glad to thank\ldots 
% \end{acknowledgementsen}
