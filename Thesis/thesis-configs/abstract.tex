\begin{abstractzh}
關於中國在「一帶一路」倡議下向接受國家提供的過度貸款是否導致這些國家陷入高度債務並最終陷入「債務陷阱」的爭議仍然存在。本論文旨在通過運用一個主權債務模型的校準,從實證角度對這個議題進行考察。具體而言,本研究聚焦於兩個在中國戰略上具有重要地位的國家 --- 斯里蘭卡和巴基斯坦。

研究結果確認了這兩個國家在接受大量貸款後確實陷入了債務違約的狀態。基於這些結果,本研究提出了兩個債務陷阱的類別,並將斯里蘭卡和巴基斯坦分別歸入不同的類別。這種分類方法為債務陷阱外交的相關文獻提供了客觀的評估和呈現方式。
\bigbreak
\noindent \textbf{關鍵字:}{\, \makeatletter \@keywordszh \makeatother}
\end{abstractzh}

\begin{abstracten}

    Ongoing debates remain surrounding whether the excessive loans provided by China under the Belt and Road Initiative lead to high indebtedness and eventual ``debt traps'' for recipient countries. This thesis empirically examines this topic through the calibration of a sovereign debt model. Specifically, the study focuses on two strategically important countries --- Sri Lanka and Pakistan.
    The research findings validate the notion that these two countries indeed fell into the default set once they received substantial loan amounts, allowing for two categories of debt traps into which Sri Lanka and Pakistan reside. This categorization offers an objective assessment and presentation method within the literature on debt-trap diplomacy.
    \bigbreak
\noindent \textbf{Keywords:}{\, \makeatletter \@keywordsen \makeatother}
\end{abstracten}

\begin{comment}
\category{I2.10}{Computing Methodologies}{Artificial Intelligence --
Vision and Scene Understanding} \category{H5.3}{Information
Systems}{Information Interfaces and Presentation (HCI) -- Web-based
Interaction.}

\terms{Design, Human factors, Performance.}

\keywords{Region of interest, Visual attention model, Web-based
games, Benchmarks.}
\end{comment}
